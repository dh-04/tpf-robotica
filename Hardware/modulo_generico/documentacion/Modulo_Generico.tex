\documentclass[a4paper,10pt]{article}

%% Paquetes Adicionales %%

\usepackage[spanish]{babel}
\selectlanguage{spanish}
\spanishdecimal{.}
\addto\captionsspanish{\def\tablename{Cuadro}}
\usepackage{fancyhdr}
\usepackage{graphics}
\usepackage[dvips]{graphicx}
\usepackage[normal]{caption2}
\usepackage{amsfonts,amssymb,amsmath,amsthm}
\usepackage[T1]{fontenc}
\usepackage{moreverb}

%% Declaracion de comandos %%

\newtheorem{lema}{Lema}
\newtheorem{teor}{Teorema}
\newtheorem{propos}{Proposici\'on}
\newtheorem{corol}{Corolario}

\newcommand{\mivec}[1]{\mathbf{#1}}
\newcommand{\vers}[1]{\mivec{\check{#1}}}
\newcommand{\deriv}[2]{\frac{\mathrm{d}#1}{\mathrm{d}#2}}
\newcommand{\expo}[1]{~10^{#1}}
\newcommand{\uni}[1]{\mathrm{#1}} 

\newcommand{\prop}[1]{\begin{propos} #1 \end{propos}}
\newcommand{\teo}[1]{\begin{teor} #1 \end{teor}}
\newcommand{\cor}[1]{\begin{corol} #1 \end{corol}}
\newcommand{\lem}[1]{\begin{lema} #1 \end{lema}}

%% Encabezado y Pie de Pagina %%

\pagestyle{plain}
\lhead{}
\chead{}
\rhead{}
\cfoot{\thepage}
\renewcommand{\footrulewidth}{0.4pt}

%%\author{Juan Ignacio Go\~ni}

\makeindex

%% Titulo %%
\begin{document}
\title{{\ Trabajo pr\'actico final \\ Robot recolector de residuos \\ Placa m\'odulo gen\'erico}}

%\date{}

%% Comienzo del documento %%

\maketitle

\begin{abstract}
En el presente se establecen las especificaciones para la placa del m\'odulo gen\'erico.
Se expone el circuito de la placa, explica funcionamiento y se muestran posibles usos.

\textbf{Palabras Clave: }\emph{Robot, residuos, protocolo, serial, rs-232, daisy chain}.
\end{abstract}

%\thispagestyle{fancy}

%% COMIENZO DEL TEXTO %%

\section{Introducci\'on}
\label{introduccion}

**TODO**

\section{Microcontrolador}
\label{microcontrolado}

El microcontrolador elegido para la placa es el 16F88 de Microchip.
Cuenta con una memoria \emph{FLASH} para 4096 instrucciones de programa, una memoria \emph{RAM} de 368 bytes y una memoria \emph{EEPROM} de 256 bytes.
En la subsecci\'on \ref{perifericos} se listan algunos de los principales perif\'ericos incluidos en el microcontrolador.
Se utiliza con un cristal externo de 20MHz como clock.

Para la carga y debug del firmware espec\'ifico para cada placa se utiliza el programador \emph{ICD2}, como se explica en la secci\'on \ref{programador}.
Cuenta con un reducido set de instrucciones b\'asicas todas con el mismo tiempo de ejecuci\'on.

\subsection{Perif\'ericos}
\label{perifericos}

El microcontrolador 16F88 cuenta con 2 puertos de 8 entradas y salidas cada uno de tipo TTL y CMOS.
Cada pin se encuentra multiplexado con uno o m\'as perif\'ericos internos.

\subsubsection{Timers}
\label{timers}

Cuenta con 3 timers o contadores.

El \emph{TMR0} es de 8 bits y contiene un \emph{preescaler} de 8bits, es usado como WDT.
Tambi\'en puede ser utilizado como contador externo por el pin \emph{RA4}.

El \emph{TMR1} es de 16 bits y contiene un \emph{preescaler} de 2bits.
Puede ser utilizado como contador externo por el pin \emph{RB6} o con un cristal externo conectado a los pines \emph{RB6} y \emph{RB7}.

El \emph{TMR2} es de 8 bits, contiene un \emph{preescaler} de 2 bits y contiene un \emph{postscaler} de 4 bits.
Es de vital importancia para el m\'odulo de PWM por hardware.

\subsubsection{ADC}
\label{adc}

Cuenta con un conversor anal\'ogico digital de 8 o 10 bits multiplexado en 7 canales, 5 canales en el puerto A y 2 en el puerto B.
Es posible definir voltajes de referencia mediante ciertos pines o usar valores internos de referencia como \emph{Vcc} y \emph{GND}.

\subsubsection{PWM}
\label{pwm}

Cuenta con un m\'odulo de generaci\'on de un PWM por hardware de 10 bits de resoluci\'on con el ciclo y per\'iodo configurable mediante el \emph{TMR2}

\subsubsection{AUSART}
\label{ausart}

Asynchronous Receiver Transmitter
(AUSART/SCI) with 9-bit address detection:

- RS-232 operation using internal oscillator
(no external crystal required)

\subsubsection{Otros}
\label{otros}

Ver el pdf de referencia...

\section{Comunicaci\'on}
\label{comunicacion}

daisy chain, rs232, conexionado, switch, logica de LAST y LINK, PINOUT!

\section{Alimentaci\'on}
\label{alimentacion}

PINOUT, voltajes

\section{Configuraci\'on}
\label{configuracion}

switches, headers de leds y pines

\section{Posibles usos}
\label{usos}

uso de los pines...

\section{Esquem\'atido}
\label{esquematico}

esquematico de la placa

\section{Circuito}
\label{circuito}

capas del circuito

\section{El programador}
\label{programador}

pines de conexionado, modelo, marca, bla

\section{C\'odigo b\'asico}
\label{codigo}

codigo minimo para que funcione con el protocolo, includes, etc...

\end{document}