\documentclass[a4paper,11pt]{article}

%% Paquetes Adicionales %%

\usepackage[spanish]{babel}
\selectlanguage{spanish}
\spanishdecimal{.}
\addto\captionsspanish{\def\tablename{Cuadro}}
\usepackage{fancyhdr}
\usepackage{graphics}
\usepackage[dvips]{graphicx}
\usepackage[normal]{caption2}
\usepackage{amsfonts,amssymb,amsmath,amsthm}
\usepackage[T1]{fontenc}
\usepackage{moreverb}


%% Declaracion de comandos %%

\newtheorem{lema}{Lema}
\newtheorem{teor}{Teorema}
\newtheorem{propos}{Proposici\'on}
\newtheorem{corol}{Corolario}


\newcommand{\mivec}[1]{\mathbf{#1}}
\newcommand{\vers}[1]{\mivec{\check{#1}}}
\newcommand{\deriv}[2]{\frac{\mathrm{d}#1}{\mathrm{d}#2}}
\newcommand{\expo}[1]{~10^{#1}}
\newcommand{\uni}[1]{\mathrm{#1}} 

\newcommand{\prop}[1]{\begin{propos} #1 \end{propos}}
\newcommand{\teo}[1]{\begin{teor} #1 \end{teor}}
\newcommand{\cor}[1]{\begin{corol} #1 \end{corol}}
\newcommand{\lem}[1]{\begin{lema} #1 \end{lema}}


%% Encabezado y Pie de Pagina %%

\pagestyle{plain}
\lhead{}
\chead{}
\rhead{}
\cfoot{\thepage}
\renewcommand{\footrulewidth}{0.4pt}

%% Titulo %%
\begin{document}
\title{{\ Trabajo pr\'actico final \\ Robot recolector de residuos \\ Protocolo de comunicaci\'on}}

%\date{}

%% Comienzo del documento %%

\maketitle
\begin{abstract}
En el presente se establece el protocolo de comunicaci\'on entre para el envio y recepci\'on de comandos hacia los controladores
de motores, servos y sensores que proveen informaci\'on del entorno al robot recolector de residuos. El protocolo est\'a dise\~nado
para ser transmitido a traves \emph{RS-232} utilizando la configuraci\'on de \emph{Daisy-Chain} entre las distintas placas controladoras.

\textbf{Palabras Clave: }\emph{Robot, residuos, protocolo, serial, rs-232, daisy chain, motor, servo, sensor, tel\'emetro, ultrasonido, distancia, bateria}.
\end{abstract}

%\thispagestyle{fancy}

%% COMIENZO DEL TEXTO %%

\section{Introducci\'on}
\label{introduccion}

TODO

\section{Formato del paquete}
\label{formato_paquete}


El paquete consta de un header com\'un con datos que identifican el emisor y receptor del paquete, el comando a enviar y posibles datos extras que sean requeridos.
Tanto los paquetes de envio de datos como los de respuesta tienen el mismo formato y comparten el valor en el campo de comando.

\begin{table}
\begin{center}
\begin{tabular}{|c|c|c|c|c|c|c|c|}
\hline
TIPO & ID & TIPO & ID & COMANDO & LARGO & DATO & CRC \\
\hline
\end{tabular}
\caption{Formato y header del paquete de datos}
\label{formato_paquete_tabla}
\end{center}
\end{table}


\subsection{TIPO}
\label{tipo_destinatario}

	El comando consta de 1 byte.
	Informa el grupo del destinatario del paquete.
	Grupos predefinidos en el listado \ref{grupos_listado}.
	Bit 7: Broadcast a todos los IDs de un mismo grupo.
	Bits 6-0: Grupo al que va dirigido el paquete. El grupo 0x7F esta reservado para hacer broadcast a todas las placas sin importar el grupo (Grupo 0xFF).

\subsection{ID}
\label{id_destinatario}

	El comando consta de 1 byte.
	Marca el n\'umero de id de la placa de destino
	Si el bit 7 del campo \emph{TIPO} de destino esta activo, el \emph{ID} se ignora.

\subsection{TIPO}
\label{tipo_emisor}

	El comando consta de 1 byte.
	Determina el grupo del emisor del paquete para la respuesta del paquete.
	Grupos predefinidos en el listado \ref{grupos_listado}.
	Bit 7: El 0 es mandatorio, la respuesta debe tener un \'unico destinatario.
	Bit 6-0: Grupo de origen del paquete.
	El grupo 0x7F esta reservado para hacer broadcast a todas las placas sin importar el grupo y no debe ser utilizado como origen del paquete.

\subsection{ID}
\label{id_emisor}

	El comando consta de 1 byte.
	Es mandatorio, debe estar informado para una correcta respuesta. En caso contrario, se toma como valor por defecto el del controlador principal, en este caso, la PC.
	Informa a que n\'umero de id de la placa debe ser dirigida la respuesta del paquete.
	
\subsection{COMANDO}
\label{comando}

	El comando consta de 1 byte.
	Comando enviado al destino, que puede o no tener datos en el campo \emph{DATO}.
	Definidos en la secci\'on \ref{comandos}.

\subsection{LARGO}
\label{largo}

	El comando consta de 1 byte.
	Largo en bytes del campo \emph{DATO}.
	En caso que el largo del campo \emph{DATO} sea cero, igual debe ser informado con 0x00.

\subsection{DATO}
\label{dato}

	El comando consta de \emph{LARGO} bytes.
	Contiene los par\'ametros o datos extras que puedan ser necesarios para el comando enviado.
	En el caso que el comando no los requiera, el campo debe ser nulo y el largo ser\'a 0x00.

\subsection{CRC}
\label{crc}

	El comando consta de 1 byte.
	C\'alculo de CRC sobre el paquete enviado.
	-Algoritmo a ser determinado-

\section{Posibles comandos}
\label{comandos}

El campo \emph{COMANDO} determina la acci\'on que debe realizarse en el destinatario o la respuesta al comando recibido.
El rango para los comandos comunes a todos los grupos de tarjetas son desde 0x00 hasta 0x3F.
Los comandos especificos para cada grupo deben ser desde 0x40 hasta 0xFF.

\begin{itemize}
	\item \emph{INITIALIZE}
	\item \emph{RESET CARD}
	\item \emph{PING}
	\item \emph{ERROR}
\label{lista_comandos}
\end{itemize}

\subsection{INIT}
\label{init}

Sincroniza el inicio de todas las placas en la cadena.
Debe ser recibido por la placa para inicializarse y poder informar al controlador principal de su existencia.

\subsubsection*{Comando enviado}
\label{init_comando_enviado}

\begin{itemize}
	\item{COMANDO:} 0x01
	\item{DATO:} vacio
\end{itemize}

\subsubsection*{Respuesta al comando}
\label{init_respuesta}

\begin{itemize}
	\item{COMANDO:} 0x01
	\item{DATO:} Descripci\'on de la placa en texto plano
\end{itemize}

\subsection{RESET}
\label{reset}

Pide el reset de la tarjeta

\subsubsection*{Comando enviado}
\label{reset_comando_enviado}

\begin{itemize}
	\item{COMANDO:} 0x02
	\item{DATO:} vacio
\end{itemize}

\subsubsection*{Respuesta al comando}
\label{reset_respuesta}

Sin respuesta

\subsection{PING}
\label{ping}

Envia un ping a la placa

\subsubsection*{Comando enviado}
\label{ping_comando_enviado}

\begin{itemize}
	\item{COMANDO:} 0x03
	\item{DATO:} vacio
\end{itemize}

\subsubsection*{Respuesta al comando}
\label{ping_respuesta}

\begin{itemize}
	\item{COMANDO:} 0x03
	\item{DATO:} vacio
\end{itemize}

\subsection{ERROR}
\label{error}

Informa que ha habido un error.

\subsubsection*{Comando enviado}
\label{error_comando_enviado}

\begin{itemize}
	\item{COMANDO:} 0x04
	\item{DATO:} 1 byte con el c\'odigo de error y la descripci\'on del error en texto plano.
\end{itemize}

\subsubsection*{Respuesta al comando}
\label{error_respuesta}

Sin respuesta.

\section{Comandos espec\'ificos}
\label{comandos_especificos}

Cada grupo de placas tiene comandos propios y espec\'ificos dependiendo de la funci\'on que deban desempe\~nar en el sistema.
Existen grupos con comandos predefinidos como se aprecia en el listado \ref{grupos_listado}.
El rango para nuevos grupos se extiende hasta el valor 0x7E.

\begin{itemize}
	\item \emph{MAIN CONTROLLER} - secci\'on \ref{grupo_main_controller}
	\item \emph{DC MOTOR} - secci\'on \ref{grupo_dc_motor}
	\item \emph{SERVO MOTOR} - secci\'on \ref{grupo_servo_motor}
	\item \emph{DISTANCE SENSOR} - secci\'on \ref{grupo_distance_sensor}
	\item \emph{FLOOR SENSOR} - secci\'on \ref{grupo_floor_sensor}
	\item \emph{ULTRASONIC SENSOR} - secci\'on \ref{grupo_ultrasonic_sensor}
	\item \emph{BATTERY CONTROLLER} - secci\'on \ref{grupo_battery_controller}
	\item \emph{TRASH BIN} - secci\'on \ref{grupo_trash_bin}
\label{grupos_listado}
\end{itemize}

\section{MAIN CONTROLLER}
\label{grupo_main_controller}

NOT YET DEFINED

\section{DC MOTOR}
\label{grupo_dc_motor}

\subsection{DIRECTION}
\label{direction}

Seteo del sentido de giro del motor

\subsubsection*{Comando enviado}
\label{direction_comando_enviado}

\begin{itemize}
	\item{COMANDO:} 0x00
	\item{DATO:} 0x00 para sentido horario \'o 0x01 para sentido anti-horario.
\end{itemize}

\subsubsection*{Respuesta al comando}
\label{direction_respuesta}

Sin respuesta

\subsection{SET SPEED}
\label{set_speed}

Seteo de la velocidad del motor en cuentas del encoder por segundo

\subsubsection*{Comando enviado}
\label{set_speed_comando_enviado}

\begin{itemize}
	\item{COMANDO:} 0x01
	\item{DATO:} consta de 6 bytes.
		SENTIDO = 0x00 para sentido horario \'o 0x01 para sentido anti-horario.
		VELOCIDAD = 0x0000 a 0xFFFF. Valores positivos, en cuentas por seguntos.
\end{itemize}

\subsubsection*{Respuesta al comando}
\label{set_speed_respuesta}

Sin respuesta

\subsection{SET ENCODER}
\label{set_encoder}

Seteo de la cantidad de cuentas historicas del encoder

\subsubsection*{Comando enviado}
\label{set_encoder_comando_enviado}

\begin{itemize}
	\item{COMANDO:} 0x02
	\item{DATO:} 0x0000 a 0xFFFF. Valores con signo, 0xFFFF es -32768 y 0x7FFF es 32767
\end{itemize}

\subsubsection*{Respuesta al comando}
\label{set_encoder_respuesta}

Sin respuesta

\subsection{GET ENCODER}
\label{get_encoder}

Obtener la cantidad de cuentas historicas del encoder

\subsubsection*{Comando enviado}
\label{get_encoder_comando_enviado}

\begin{itemize}
	\item{COMANDO:} 0x03
	\item{DATO:} vacio
\end{itemize}

\subsubsection*{Respuesta al comando}
\label{get_encoder_respuesta}

\begin{itemize}
	\item{COMANDO:} 0x03
	\item{DATO:} 0x0000 a 0xFFFF. Valores con signo, 0xFFFF es -32768 y 0x7FFF es 32767
\end{itemize}

\subsection{RESET ENCODER}
\label{reset_encoder}

Resetear las cuentas historicas a cero

\subsubsection*{Comando enviado}
\label{reset_encoder_comando_enviado}

\begin{itemize}
	\item{COMANDO:} 0x04
	\item{DATO:} vacio
\end{itemize}

\subsubsection*{Respuesta al comando}
\label{reset_encoder_respuesta}

Sin respuesta

\subsection{SET ENCODER TO STOP}
\label{set_encoder_to_stop}

Seteo de cuantas cuentas a girar hasta detenerse

\subsubsection*{Comando enviado}
\label{set_encoder_to_stop_comando_enviado}

0x0000 a 0xFFFF. Valores con signo, 0xFFFF es -32768 y 0x7FFF es 32767

\begin{itemize}
	\item{COMANDO:} 0x05
	\item{DATO:} vacio
\end{itemize}

\subsubsection*{Respuesta al comando}
\label{set_encoder_to_stop_respuesta}

Sin respuesta

\subsection{GET ENCODER TO STOP}
\label{get_encoder_to_stop}

Obtener la cantidad de cuantas cuentas restantes hasta detenerse

\subsubsection*{Comando enviado}
\label{get_encoder_to_stop_comando_enviado}

\begin{itemize}
	\item{COMANDO:} 0x06
	\item{DATO:} vacio
\end{itemize}

\subsubsection*{Respuesta al comando}
\label{get_encoder_to_stop_respuesta}

\begin{itemize}
	\item{COMANDO:} 0x06
	\item{DATO:} 0x0000 a 0xFFFF. Valores con signo, 0xFFFF es -32768 y 0x7FFF es 32767
\end{itemize}

\subsection{DONT STOP}
\label{dont_stop}

Borrar la limitacion de cuentas para frenar

\subsubsection*{Comando enviado}
\label{dont_stop_comando_enviado}

\begin{itemize}
	\item{COMANDO:} 0x06
	\item{DATO:} vacio
\end{itemize}

\subsubsection*{Respuesta al comando}
\label{dont_stop_respuesta}

Sin respuesta


\section{SERVO MOTOR}
\label{grupo_servo_motor}

\section{DISTANCE SENSOR} 
\label{grupo_distance_sensor}

\section{FLOOR SENSOR} 
\label{grupo_floor_sensor}

\section{ULTRASONIC SENSOR} 
\label{grupo_ultrasonic_sensor}

\section{BATTERY CONTROLLER} 
\label{grupo_battery_controller}

\section{TRASH BIN} 
\label{grupo_trash_bin}

\section{Ejemplos}
\label{ejemplos}

\section{Conclisi\'on}
\label{conclusion}

\end{document}