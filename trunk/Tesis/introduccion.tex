
\section{\label{intro} Introducci\'on}
El tratamiento de los residuos es uno de los problemas principales que 
aquejan a las grandes ciudades. Entre otros problemas, la basura no 
recolectada puede da\~nar al medio ambiente de diferentes maneras. Recipientes
abiertos como vasos o botellas pueden acumular agua de lluvia favoreciendo la 
proliferaci\'on de mosquitos que pueden causar enfermedades como el dengue. Los
residuos no recolectados pueden ser acarreados por los caudales de lluvia
perjudicando los sistemas de desag\"ues y contaminando los r\'ios y mares donde
finalmente desembocan. Esta problem\'atica es tratada por distintos organismos
y sectores de la sociedad, tanto gubernamentales como organizaciones sin fines
de lucro. El objetivo de este trabajo es contribuir a esta \'area desde el
punto de vista de la rob\'otica.
\\\indent
Con este prop\'osito, dise\~namos un prototipo de robot m\'ovil capaz de
reconocer y recolectar residuos en un ambiente din\'amico como lo es la terraza
de nuestra universidad. Este robot, navega a trav\'es de dicho entorno
utilizando un set de ruedas en b\'usqueda de elementos residuales que
son detectados mediante una c\'amara a bordo. El robot cuenta adem\'as,
con un sistema de comportamientos que le permite recorrer el entorno
eficazmente y recargar su bater\'ia cuando sea necesario, esquivando
los obst\'aculos que se le pudieran presentar en el camino. El reconocimiento
de residuos se logra a trav\'es de un sistema de visi\'on que realiza el
procesamiento de las im\'agenes capturadas por la c\'amara en busca de colillas
de cigarrillos, vasos y platos. Para realizar esto, equipamos al robot con
una serie de sensores y dispositivos de procesamiento que permiten la
ejecuci\'on de estos algoritmos y rutinas programadas para dicha labor.
\\\indent
Lo que resta del trabajo lo dividimos de la siguiente forma: en la secci\'on
\ref{h_hardware} explicamos detalladamente las caracter\'isticas f\'isicas del
robot, desde su composici\'on hasta su ensamblaje. En la secci\'on
\ref{sec:behaviour_architecture_all} mostramos la implementaci\'on y
arquitectura de los comportamientos del robot que definen la manera en que
\'este interact\'ua con el entorno. En la secci\'on \ref{sec:vision}
detallamos la implementaci\'on del algoritmo de procesamiento de im\'agenes
utilizado para el reconocimiento de residuos. La secci\'on
\ref{sec:conclusiontotal} concluye el trabajo.

