
\section{Conclusiones}
En este trabajo hemos desarollado la construcci\'on e implementaci\'on
de un robot autonomo, movil, capaz de recolectar residuos en un entorno din\'amico. 
Como mecanismo de locomoci\'on utilizamos dos ruedas de tracci\'on y una 
tercer rueda de tipo castor para proveer equilibrio, movilizadas por un 
par de motores de corriente contin\'ua provistos de una caja reductora y encoders. 
Para el sensado del entorno equipamos al robot con sensores 
infrarrojos, sensores de distancia por ultrasonido, sensores reflectivos 
de piso y un medidor de tensi\'on en la bater\'ia. Para cada uno de estos 
dispositivos desarrollamos circuitos de control y rutinas que nos permitieron 
la obtenci\'on de datos como as\'i la interacci\'on con el entorno. 
Los actuadores y sensores son controlados desde una computadora a bordo de tipo
netbook que utilizamos como unidad principal de procesamiento. 
Realizamos el interconexionado de las placas de control con el computador
utilizando una red del tipo daisy-chain en donde la informaci\'on se transmite 
a trav\'es de un protocolo escalable capaz de conectar distintos tipos de 
placas entre si y con el computador principal.\\
\indent  Implementamos la interacci\'on del 
robot con su entorno mediante la definici\'on de distintos comportamientos dispuestos 
bajo un arquitectura de tipo subsumption, los mas importantes son: 
wandering, evitamiento de obst\'aculos, enfocar residuos, seguimiento de 
l\'inea y recolectar basura. Para la navegaci\'on, desarollamos
un sistema de odometr\'ia que permite estimar la posici\'on del robot en 
el entorno a trav\'es del uso de los encoders de los motores.
Para el reconocimiento de residuos, instalamos una c\'amara de tipo webcam 
conectada al computador principal donde corre un sistema de visi\'on que 
diseñamos para el reconocimiento de colillas de cigarrillo, vasos de pl\'astico 
y platos descartables. \'Este sistema esta compuesto por una etapa de 
pre-procesamiento que consiste en un filtro de color, operaciones morfol\'ogicas 
y operaciones de tipo umbral para la elimnaci\'on de ruido y una seg\'unda 
etapa de an\'alisis de contornos en donde el sistema verifica que las figuras encontradas 
correspondan con los residuos buscados. Agregamos a este sistema de 
visi\'on un m\'odulo de predicci\'on y un m\'odulo de focalizaci\'on por 
el cual  aumentamos la confianza y eficiencia del algoritmo. \\ 
\indent Realizamos la simulaci\'on de los comportamientos del robot en el 
ambiente webots en  
donde se utilizaron versiones virtuales de los sensores, actuadores y entorno 
para verificar su buen funcionamiento. \\
\indent Como posibles extensiones destacamos .... 
ACA NOSE SI PONER LAS POSIBLES EXTENSIONES QUE YA MENCIONAMOS O PONER ALGUNAS NUEVAS.
\label{conc}
