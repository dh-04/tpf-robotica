
\section{protocolo de comunicacion}
\label{hA_protocolo}

Protocolo...

\newpage

\section{C\'odigo fuente}
\label{hA_source}

En este apartado mostramos el c\'odigo fuente usado para cada una de las placas controladoras
del proyecto.
Peque\~nos cambios existen en \emph{ID} de grupo y placa fueron necesarios para poder que sean
\'unicas dentro de la cadena de comunicaci\'on.

\subsection{Placa gen\'erica}
\label{hA_source_gen}

A la placa gen\'erica la pensamos para testeo y como punto de partida de nuevas
implementaciones, agregamos el c\'odigo que usamos como base para la programaci\'on de las otras
placas.

{\scriptsize
\begin{verbatimtab}
#define CARD_GROUP	MOTOR_DC	// Ver protocol.h
#define CARD_ID		0		// Valor entre 0 y E

// Descripcion de la placa
#define DESC		"PLACA GENERICA - 1.0" // Maximo DATA_SIZE bytes

/* Modulo Generico - main.c
 * PIC16F88 - MAX232 - GENERICO
 *
 *                               PIC16F88
 *                .------------------------------------.
 *               -|RA2/AN2/CVREF/VREF           RA1/AN1|- LED
 *               -|RA3/AN3/VREF+/C1OUT          RA0/AN0|- 
 *           LED -|RA4/AN4/T0CKI/C2OUT    RA7/OSC1/CLKI|- XT CLOCK pin1, 27pF to GND
 * RST/ICD2:MCLR -|RA5/MCLR/VPP           RA6/OSC2/CLKO|- XT CLOCK pin2, 27pF to GND
 *           GND -|VSS                              VDD|- +5v
 *               -|RB0/INT/CCP1       RB7/AN6/PGD/T1OSI|- ICD2:PGD/
 *               -|RB1/SDI/SDA  RB6/AN5/PGC/T1OSO/T1CKI|- ICD2:PGC/
 *  MAX232:R1OUT -|RB2/SDO/RX/DT           RB5/SS/TX/CK|- MAX232:T1IN
 *               -|RB3/PGM/CCP1             RB4/SCK/SCL|- 
 *                '------------------------------------'
 *    
 */

#include <16F88.h>
#DEVICE ADC = 10
#include <stdio.h>
#include <string.h>
#include <stdlib.h>

#fuses HS,NOWDT,NOPROTECT,NOLVP
#use delay (clock=20000000)

#use rs232(BAUD=115200,PARITY=N,XMIT=PIN_B5,RCV=PIN_B2,BITS=8,ERRORS,TIMEOUT=1,STOP=1,UART1)
#use fast_io(A)
#use fast_io(B)

#byte porta=0x05
#byte portb=0x06

// Led
#bit led1=porta.1
#bit led2=porta.4

// MAX232
#bit tx=portb.5
#bit rx=portb.2

#include <../../protocolo/src/protocol.c>
/*
** Variables definidas en protocol.c

short reset; // Variable para hacer el reset
short crcOK; // Informa si el CRC del paquete parseado fue correcto
short sendResponse; // Informa que no debe mandarse la respuesta automatica

char buffer[MAX_BUFFER_SIZE];	// Buffer de recepcion de comandos
int buffer_write;		// Indice de escritura
int buffer_read;		// Indice de lectura
int data_length;		// Largo de los datos en el buffer

struct command_t command; 	// Comando parseado
struct command_t response; 	// Respuesta

** Implementar las siguientes funciones (usadas por el protocolo)

void init(); // Inicializa puertos y variables
void doCommand(struct command_t * cmd); // Examina y ejecula el comando

***/

void init()
{
	// Inicializa puertos
	set_tris_a(0b11100101);
	set_tris_b(0b11100110);

	// Variable para hacer el reset
	reset = false;

	return;	
}	

void main()
{
	// Placa Generica - Implementacion del protocolo
	init();

	// Init del protocol
	initProtocol();

	// FOREVER
	while(true)
	{
		// Hace sus funciones...

		// Protocolo
		runProtocol(&command);
	}

	return;
}

/* Verifica que el comando sea valido y lo ejecuta */
void doCommand(struct command_t * cmd)
{
	int crc, i, len;
		
	// Calculo del CRC
	crc = generate_8bit_crc((char *)cmd, cmd->len, CRC_PATTERN);
	
	// CRC ok?
	if (cmd->crc != crc)
	{		
		// Creo respuesta de error
		response.len = MIN_LENGTH + cmd->len + 2 + 1;
		response.to = cmd->from;
		response.from = THIS_CARD;
		response.cmd = COMMON_ERROR;
		response.data[0] = 0x00;
		// Agrego el paquete que contiene el error de CRC
		response.data[1] = cmd->len;
		response.data[2] = cmd->to;
		response.data[3] = cmd->from;
		response.data[4] = cmd->cmd;
		// Campo data
		len = cmd->len - MIN_LENGTH;
		for (i = 0; i < len; i++)
			response.data[5 + i] = (cmd->data)[i];
		// CRC erroneo
		response.data[5 + len] = cmd->crc;
		// CRC esperado
		response.data[5 + len + 1] = crc;
		// CRC de la respuesta
		response.crc = generate_8bit_crc((char *)(&response), response.len, CRC_PATTERN);
	
		crcOK = false;
		return;
	}

	crcOK = true;
	
	// Minimo todos setean esto
	response.len = MIN_LENGTH;
	response.to = cmd->from & 0x77;
	response.from = THIS_CARD;
	response.cmd = cmd->cmd | 0x80;

	switch (cmd->cmd)
	{
		// Comandos comunes
		case COMMON_INIT: 
			init();
			// Enviar la descripcion de la placa en texto plano
			strcpy(response.data, DESC);
			response.len += strlen(response.data);
		break;
		case COMMON_RESET: 
			// Enviar la descripcion de la placa en texto plano
			strcpy(response.data, DESC);
			response.len += strlen(response.data);
			// Reset!
			reset = true;
		break;
		case COMMON_PING: 
			// No hace falta hacer mas nada
		break;
 		case COMMON_ERROR:
			// Por ahora se ignora el comando
		break;
		
		/* Comandos especificos */

 		case 0x40:
			/* 
			:DATO:
			-
			:RESP:
			-
			*/
		break;

		default:
			response.len++;
			response.cmd = COMMON_ERROR;
			response.data[0] = 0x01; // Comando desconocido
		break;
	}	

	// CRC de la respuesta
	response.crc = generate_8bit_crc((char *)(&response), response.len, CRC_PATTERN);

	return;
}
\end{verbatimtab}
}

\subsection{Archivo \emph{protocolo.c}}
\label{hA_source_protocolo}

C\'odigo fuente com\'un que genera un buffer c\'iclico con los paquetes entrantes, parsea el
buffer y llama a la funci\'on que analiza al paquete seg\'un la implementaci\'on propia para
cada placa.

{\scriptsize
\begin{verbatimtab}
#include <../../protocolo/src/protocol.h>

short reset;
short crcOK;
short sendResponse;

char buffer[MAX_BUFFER_SIZE];
int buffer_write;
int buffer_read;
int data_length;

// Comando parseado
struct command_t command;
// Respuesta
struct command_t response;

// Interrupcion del RS232
#INT_RDA
void RS232()
{
	disable_interrupts(INT_RDA);
	// Un nuevo dato...
	buffer[buffer_write++] = getc();
	data_length++;
	if (buffer_write == MAX_BUFFER_SIZE)
		buffer_write -= MAX_BUFFER_SIZE;
	enable_interrupts(INT_RDA);
	return;
}

/* Envia los datos por el pto serial */
void initProtocol()
{
	// Variables de comunicacion
	buffer_write = 0;
	buffer_read = 0;
	data_length = 0;
	crcOK = false;
	
	// Interrupcion Rcv
	enable_interrupts(INT_RDA);

	// Habilito las interrupciones
	enable_interrupts(GLOBAL);
}

/* Envia los datos por el pto serial */
void send(struct command_t * cmd)
{
	int i, len;
	
	len = cmd->len - 4;
	putc(cmd->len);
	putc(cmd->to);
	putc(cmd->from);
	putc(cmd->cmd);
	
	for (i = 0; i < len; i++)
	{
		putc((cmd->data)[i]);
	}
	
	// Enviar el CRC
	putc(cmd->crc);
	
	return;	
}	

void runProtocol(struct command_t * cmd)
{
	// Analiza el buffer
	if (buffer[buffer_read] < data_length)
	{
		data_length -= buffer[buffer_read] + 1;
	
		cmd->len = buffer[buffer_read++];
	
		if (buffer_read == MAX_BUFFER_SIZE)
			buffer_read = 0;
	
		cmd->to = buffer[buffer_read++];
	
		if (buffer_read == MAX_BUFFER_SIZE)
			buffer_read = 0;
	
		cmd->from = buffer[buffer_read++];
	
		if (buffer_read == MAX_BUFFER_SIZE)
			buffer_read = 0;
	
		cmd->cmd = buffer[buffer_read++];
	
		if (buffer_read == MAX_BUFFER_SIZE)
			buffer_read = 0;
	
		// Obtiene el campo DATA
		if ((buffer_read + cmd->len - MIN_LENGTH) > MAX_BUFFER_SIZE)
		{
			// DATA esta partido en el buffer ciclico
			memcpy(cmd->data, buffer + buffer_read, MAX_BUFFER_SIZE - buffer_read);
			memcpy(cmd->data + MAX_BUFFER_SIZE - buffer_read, buffer, 
				cmd->len - MIN_LENGTH - MAX_BUFFER_SIZE + buffer_read);
		} else {
			// DATA esta continuo
			memcpy(cmd->data, buffer + buffer_read, cmd->len - MIN_LENGTH);
		}
	
		buffer_read += cmd->len - MIN_LENGTH;
		if (buffer_read >= MAX_BUFFER_SIZE)
			buffer_read -= MAX_BUFFER_SIZE;
	
		cmd->crc = buffer[buffer_read++];
	
		if (buffer_read == MAX_BUFFER_SIZE)
			buffer_read = 0;

		sendResponse = true;

		// Soy el destinatario?
		if (cmd->to == THIS_CARD)
		{
			// Ejecuta el comando
			doCommand(cmd);
		} else // Es broadcast?
			if ((cmd->to & 0xF0) == 0xF0)
		{
			// Ejecuta el comando
			doCommand(cmd);

			if (crcOK == true)
			{
				// Envia la respuesta
				send(&response);
				// Envia nuevamente el comando recibido
				response = *cmd;
			}
		} else // Es broadcast para mi grupo? 
			if (((cmd->to & 0x0F) == 0x0F) &&
			 	((cmd->to & 0xF0) == THIS_GROUP))
		{
			// Ejecuta el comando
			doCommand(cmd);	
#ifdef RESEND_GROUP_BROADCAST
			if (crcOK == true)
			{
				// Envia la respuesta
				send(&response);
				// Envia nuevamente el comando recibido
				response = *cmd;
			}
#endif
		} else {
			response = *cmd;
		}
	
		// Envia la respuesta?
		if (sendResponse == true)
		{
			send(&response);
		}	

	}
	
	// Reset del micro
	if (reset == true)
	{
		reset_cpu();
	}	

}

int generate_8bit_crc(char* data, int length, int pattern)
{
	// TODO: reemplazar por el crc?
	
	int crc_byte, i;

	crc_byte = data[0];
			
	for (i = 1; i < length; i++)
		crc_byte ^= data[i];
	
	return crc_byte;
}
\end{verbatimtab}
}

\subsection{Placa controladora de motor de dc}
\label{hA_source_dc}

C\'odigo fuente de la placa controladora de motor de corriente cont\'inua.

{\scriptsize
\begin{verbatimtab}
//CCS PCM V4.023 COMPILER

#define CARD_GROUP	MOTOR_DC	// Ver protocol.h
#define CARD_ID		0		// Valor entre 0 y E

// Descripcion de la placa
#define DESC		"CONTROL MOTOR DC 1.0" // Maximo DATA_SIZE bytes

/* Modulo Motor - main.c
 * PIC16F88 - MAX232 - L298 - MR-2-60-FA
 *
 *                               PIC16F88
 *                .------------------------------------.
 *          VREF -|RA2/AN2/CVREF/VREF           RA1/AN1|- MOTOR:CHA_B
 *           LED -|RA3/AN3/VREF+/C1OUT          RA0/AN0|- L298:SEN
 *           LED -|RA4/AN4/T0CKI/C2OUT    RA7/OSC1/CLKI|- XT CLOCK pin1, 27pF to GND
 * RST/ICD2:MCLR -|RA5/MCLR/VPP           RA6/OSC2/CLKO|- XT CLOCK pin2, 27pF to GND
 *           GND -|VSS                              VDD|- +5v
 *   L298:ENABLE -|RB0/INT/CCP1       RB7/AN6/PGD/T1OSI|- ICD2:PGD
 *     MOTOR:IDX -|RB1/SDI/SDA  RB6/AN5/PGC/T1OSO/T1CKI|- ICD2:PGC/MOTOR:CHA_A
 *  MAX232:R1OUT -|RB2/SDO/RX/DT           RB5/SS/TX/CK|- MAX232:T1IN
 * L298:INPUT_B/ -|RB3/PGM/CCP1             RB4/SCK/SCL|- L298:INPUT_A
 *     ICD2:PGM   '------------------------------------'
 *    
 */

#include <16F88.h>
#DEVICE ADC = 10
#include <stdio.h>
#include <string.h>
#include <stdlib.h>

#fuses HS,NOWDT,NOPROTECT,NOLVP
#use delay (clock=20000000)

#use rs232(BAUD=115200,PARITY=N,XMIT=PIN_B5,RCV=PIN_B2,BITS=8,ERRORS,TIMEOUT=1,STOP=1,UART1)
#use fast_io(A)
#use fast_io(B)

#byte porta=0x05
#byte portb=0x06

// MAX232
#bit tx=portb.5
#bit rx=portb.2

// Led
#bit led1=porta.3
#bit led2=porta.4

// L298
#bit inputA=portb.4
#bit inputB=portb.3
#bit enable=portb.0
#bit sensor=porta.0

// Motor inuts
#bit motorIDX=portb.1
#bit channelA=portb.6
#bit channelB=porta.1

#include <../../protocolo/src/protocol.c>
/*
** Variables definidas en protocol.c

short reset; // Variable para hacer el reset
short crcOK; // Informa si el CRC del paquete parseado fue correcto
short sendResponse; // Informa que no debe mandarse la respuesta automatica

char buffer[MAX_BUFFER_SIZE];	// Buffer de recepcion de comandos
int buffer_write;				// Indice de escritura
int buffer_read;				// Indice de lectura
int data_length;				// Largo de los datos en el buffer

struct command_t command; 	// Comando parseado
struct command_t response; 	// Respuesta

** Implementar las siguientes funciones (usadas por el protocolo)

void init(); // Inicializa puertos y variables
void doCommand(struct command_t * cmd); // Examina y ejecula el comando

***/

#define MAX_CONSUMPTION	150
#define MAX_CONSUMPTION_COUNT	5

// Correcion de la cantidad de cuentas por segundo en base al periodo del TMR0
#define INTERVAL_CORRECTION	5

// Girar -> clockwise or unclockwise
// Intercambiar entre el motor derecho y el izquierdo
#define CLOCKWISE	1
#define UNCLOCKWISE	-1

// Sentido de giro del motor
signed int turn;
// Cantidad de overflows del TMR0
long tmr0_ticks;
// Valor acumulado del ADC - Consumo aprox
long adc_value;
// Ultimo valor del consumo
long last_consumption;
// Valor de duty del PWM
signed long duty;
// Cantidad de cuentas del encoder medidas por intervalo
signed long counts_real;
// Cantidad de cuentas del encoder esperadas por intervalo
signed long counts_expected;
// Cantidad de cuentas del encoder historicas (32 bits)
signed int32 counts_total;
// Cantidad de cuentas del encoder restantes para deterner el motor
signed long counts_to_stop;
// Cantidad de cuentas del encoder desde el ultimo intevalo
signed long last_counts;
signed long last_counts2;
// Tengo una cantidad de cuentas para hacer?
short counts_check;
// Corrijo el PWM segun lo esperado?
short correct_duty;
// Indica que hay que enviar una alarma de consumo
short consumption_alarm;
// Cuenta cuantas alarmas se enviaron
int alarm_count;
// Indica que hay que avisar que el motor se detuvo
short shutdown_alarm;
// Indica si se apagaron los motores por el alto consumo
short motor_shutdown;

// Variables temporales
signed int32 * tmp32;
signed long * tmp16;

/* Setea el PWM */
void SetPWM(signed long pwm);

// Interrupcion del Timer0
#INT_RTCC
void Timer0_INT()
{
	// Seteo el valor para que interrumpa cada 6.25ms
	set_timer0(12);
	
	// Comienza la lectura del ADC
	read_adc(ADC_START_ONLY);
	
	// Agrego al historico de cuentas el ultimo acumulado
	counts_real = get_timer1();
	counts_total += (counts_real - last_counts2) * turn;
	last_counts2 = counts_real;
	
	// Tengo una cantidad de cuentas para hacer?
	if (counts_check == 1)
	{
		// Verifico si pasaron las cuentas que se habian pedido
		if (counts_to_stop < 1)
		{
			// Detengo el motor
			counts_check = 0;
			counts_to_stop = 0;
			counts_expected = 0;
			duty = 0;
			SetPWM(duty);
			correct_duty = 0;
			last_counts = 0;
		} else {
			counts_real = get_timer1();
			counts_to_stop -= (counts_real - last_counts);// * turn;
			correct_duty = 1;
			last_counts = counts_real;
		}
	} else {
		correct_duty = 1;
	}
	
	// Tomo la muestra
	adc_value += read_adc(ADC_READ_ONLY);
	
	if (++tmr0_ticks == 32)
	{
		// Entra cada 200ms

		// Obtengo la cantidad de cuentas desde la ultima entrada
		counts_real = get_timer1();
		set_timer1(0);
		last_counts = 0;
		last_counts2 = 0;
		// Promedio el consumo segun la cantidad de tmr0_ticks
		last_consumption = adc_value / tmr0_ticks;
		
		if (last_consumption >= MAX_CONSUMPTION)
		{
			consumption_alarm = 1;
			if (alarm_count++ == MAX_CONSUMPTION_COUNT)
			{
				motor_shutdown = 1;
				alarm_count = 0;
				consumption_alarm = 0;
				shutdown_alarm = 1;
			}	
		}
		
		tmr0_ticks = 0;

		// Mantengo el consumo promedio desde que arranque y borro el temporal
		adc_value = 0;

		// Corrijo el PWM segun lo esperado
		if ((correct_duty == 1) && (counts_real != counts_expected))
		{
			duty += (counts_expected - counts_real) * 5;
			if (duty > 1023L)
				duty = 1023;
			else if (duty < 0)
				duty = 0;
			SetPWM(duty * turn);
		} else if ((counts_expected == 0) && (duty != 0)) {		
			SetPWM(duty = 0);
		}	
	}
	return;
}

void init()
{
	// Inicializa puertos
	set_tris_a(0b11100111);
	set_tris_b(0b11100110);

	// ***ADC***
	setup_port_a(sAN0);//|VSS_VREF);
	setup_adc(ADC_CLOCK_INTERNAL);
	set_adc_channel(0);
	setup_adc_ports(sAN0);
	// Deberia usar VREF_A2... probar
	setup_vref(VREF_HIGH | 8); // VREF a 2.5V -> no hay cambios...
	
	// ***PWM***
	setup_ccp1(CCP_PWM);
	// Seteo al PWM con f: 4.88 kHz, duty = 0
	set_pwm1_duty(0);
	setup_timer_2(T2_DIV_BY_4, 255, 1);

	// ***TIMER1 - ENCODER COUNTER***
	// Seteo el Timer1 como fuente externa y sin divisor
	setup_timer_1(T1_EXTERNAL | T1_DIV_BY_1);
	set_timer1(0);

	// ***TIMER0 - TIME BASE***
	// Seteo el Timer0 como clock -> dt = 6.25ms
	setup_counters(RTCC_INTERNAL, RTCC_DIV_128);
	set_timer0(12);
	// Interrupcion sobre el Timer0
	enable_interrupts(INT_RTCC);

	// Habilito las interrupciones
	enable_interrupts(GLOBAL);
	
	// Variable para hacer el reset
	reset = false;

	// Sentido de giro del motor
	turn = CLOCKWISE;
	// Cantidad de overflows del TMR0
	tmr0_ticks = 0;
	// Valor acumulado del ADC - Consumo aprox
	adc_value = 0;
	last_consumption = 0;
	// Valor de duty del PWM
	duty = 0;
	// Cantidad de cuentas del encoder esperadas por intervalo
	counts_expected = 0;
	// Cantidad de cuentas del encoder totales
	counts_total = 0;
	// Cantidad de cuentas del encoder restantes para deterner el motor
	counts_to_stop = 0;
	// Cantidad de cuentas del encoder desde el ultimo intevalo
	last_counts = 0;
	last_counts2 = 0;
	// Tengo una cantidad de cuentas para hacer?
	counts_check = 0;
	// Corrijo el PWM segun lo esperado?
	correct_duty = 1;
	// Indica que hay que enviar una alarma de consumo
	consumption_alarm = 0;
	// Cuenta cuantas alarmas se enviaron
	alarm_count = 0;
	// Indica que hay que enviar una alarma de consumo
	shutdown_alarm = 0;
	// Indica si se apagaron los motores por el alto consumo
	motor_shutdown = 0;

	return;	
}	

/* Setea el duty del PWM segun el valor. Positivo o negativo determina el sentido */
void setPWM(signed long pwm)
{
	if (pwm < 0)
	{
		inputA = 0;
		inputB = 1;
	} else {
		inputA = 1;
		inputB = 0;
	}
	
	if (motor_shutdown == 1)
	{
		pwm = 0;
	} else {
		pwm = (abs(pwm));
		
		if (pwm > 1023L)
			pwm = 1023;
	}

	set_pwm1_duty(pwm);
	
	return;
}

void main()
{
	// Placa Generica - Implementacion del protocolo
	init();

	// Init del protocol
	initProtocol();

counts_expected = 30;

	// FOREVER
	while(true)
	{
		// Hace sus funciones -> interrupcion
	
		// Enviar alarma de alto consumo
		if (consumption_alarm == 1)
		{
			/* Indica al controlador principal que hay un consumo extremo en el motor,
			posiblemente un atasco del motor o de la rueda.
			:DATO:
			Numero entero positivo de 16 bits en el rango desde 0x0000 hasta
			0x03FF, que representa el consumo ante el que sono la alarma.
			:RESP:
			-
			*/
			command.len = MIN_LENGTH + 2;
			command.to = MAIN_CONTROLLER;
			command.from = THIS_CARD;
			command.cmd = DC_MOTOR_MOTOR_STRESS_ALARM;
			// A la posicion 0 dentro de command.data la tomo como signed long *
			tmp16 = (command.data);
			// Le asigno el valor del ultimo consumo del motor
			(*tmp16) = last_consumption;
			command.crc = generate_8bit_crc((char *)(&command), command.len, CRC_PATTERN);
			consumption_alarm = 0;
			// Envio del comando
			send(&command);
		}
		
		// Enviar aviso de motor apagado
		if (shutdown_alarm == 1)
		{
			/* Indica al controlador principal que el motor ha sido apagado debido al alto
			consumo. Enviado luego de sucesivos avisos del comando DC_MOTOR_MOTOR_STRESS_ALARM.
			:DATO:
			Numero entero positivo de 16 bits en el rango desde 0x0000 hasta
			0x03FF, que representa el consumo ante el que sono la alarma.
			:RESP:
			-
			*/
			command.len = MIN_LENGTH + 2;
			command.to = MAIN_CONTROLLER;
			command.from = THIS_CARD;
			command.cmd = DC_MOTOR_MOTOR_SHUT_DOWN_ALARM;
			// A la posicion 0 dentro de response->data la tomo como signed long *
			tmp16 = (command.data);
			// Le asigno el valor del ultimo consumo del motor
			(*tmp16) = last_consumption;
			command.crc = generate_8bit_crc((char *)(&command), command.len, CRC_PATTERN);
			shutdown_alarm = 0;
			// Envio del comando
			send(&command);
		}
			
		// Protocolo
		runProtocol(&command);
	}

	return;
}

/* Verifica que el comando sea valido y lo ejecuta */
void doCommand(struct command_t * cmd)
{
	int crc, i, len;
		
	// Calculo del CRC
	crc = generate_8bit_crc((char *)cmd, cmd->len, CRC_PATTERN);
	
	// CRC ok?
	if (cmd->crc != crc)
	{		
		// Creo respuesta de error
		response.len = MIN_LENGTH + cmd->len + 2 + 1;
		response.to = cmd->from;
		response.from = THIS_CARD;
		response.cmd = COMMON_ERROR;
		response.data[0] = 0x00;
		// Agrego el paquete que contiene el error de CRC
		response.data[1] = cmd->len;
		response.data[2] = cmd->to;
		response.data[3] = cmd->from;
		response.data[4] = cmd->cmd;
		// Campo data
		len = cmd->len - MIN_LENGTH;
		for (i = 0; i < len; i++)
			response.data[5 + i] = (cmd->data)[i];
		// CRC erroneo
		response.data[5 + len] = cmd->crc;
		// CRC esperado
		response.data[5 + len + 1] = crc;
		// CRC de la respuesta
		response.crc = generate_8bit_crc((char *)(&response), response.len, CRC_PATTERN);
	
		crcOK = false;
		return;
	}

	crcOK = true;
	
	// Minimo todos setean esto
	response.len = MIN_LENGTH;
	response.to = cmd->from & 0x77;
	response.from = THIS_CARD;
	response.cmd = cmd->cmd | 0x80;

	switch (cmd->cmd)
	{
		// Comandos comunes
		case COMMON_INIT: 
			init();
			// Enviar la descripcion de la placa en texto plano
			strcpy(response.data, DESC);
			response.len += strlen(response.data);
		break;
		case COMMON_RESET: 
			// Enviar la descripcion de la placa en texto plano
			strcpy(response.data, DESC);
			response.len += strlen(response.data);
			// Reset!
			reset = true;
		break;
		case COMMON_PING: 
			// No hace falta hacer mas nada
		break;
 		case COMMON_ERROR:
			// Por ahora se ignora el comando
		break;
		
		/* Comandos especificos */

 		case DC_MOTOR_SET_DIRECTION:
			/* Seteo del sentido de giro del motor
			:DATO:
			0x00 para sentido horario o 0x01 para sentido anti-horario.
			:RESP:
			-
			*/
			if (((cmd->data)[0] & 0x01) == 0)
			{
				turn = CLOCKWISE;
			} else {
				turn = UNCLOCKWISE;
			}
		break;
 		case DC_MOTOR_SET_DC_SPEED:
			/* Seteo de la velocidad del motor en cuentas del encoder por segundo
			:DATO:
			0x00 para sentido horario o 0x01 para sentido anti-horario. Numero 
			entero de 16 bits con signo, que representa la velocidad en cuentas 
			por segundos.
			:RESP:
			-
			*/
			if (((cmd->data)[0] & 0x01) == 0)
			{
				turn = CLOCKWISE;
			} else {
				turn = UNCLOCKWISE;
			}
			// A la posicion 1 dentro de cmd->data la tomo como signed long *
			tmp16 = (cmd->data) + 1;
			// Le asigno el valor de la velocidad ajustada a 1 segundo
			counts_expected = (*tmp16) / INTERVAL_CORRECTION;
			// Habilita el motor
			motor_shutdown = 0;
		break;
 		case DC_MOTOR_SET_ENCODER:
			/* Seteo de la cantidad de cuentas historicas del encoder
			:DATO:
			Numero entero de 32 bits con signo, con el valor para setear en el
			historico del encoder.
			:RESP:
			-
			*/
			// A la posicion 0 dentro de cmd->data la tomo como signed int32 *
			tmp32 = (cmd->data);
			// Le asigno el valor de las cuentas historicas
			counts_total = (*tmp32);
		break;
 		case DC_MOTOR_GET_ENCODER:
			/* Obtener la cantidad de cuentas historicas del encoder
			:DATO:
			-
			:RESP:
			Numero entero de 32 bits con signo, que representa el valor historico
			del encoder.
			*/
			// A la posicion 0 dentro de response->data la tomo como signed int32 *
			tmp32 = (response.data);
			// Le asigno el valor de las cuentas historicas
			(*tmp32) = counts_total;
			// Corrijo el largo del paquete
			response.len += 4;
		break;
 		case DC_MOTOR_RESET_ENCODER:
			/* Resetear las cuentas historicas a cero
			:DATO:
			-
			:RESP:
			-
			*/
			counts_total = 0;
		break;
 		case DC_MOTOR_SET_ENCODER_TO_STOP:
			/* Seteo de cuantas cuentas debe girar hasta detenerse
			:DATO:
			Numero entero de 16 bits con signo, que representa la cantidad de
			cuentas del encoder restantes para que el motor se detenga.
			:RESP:
			-
			*/
			// A la posicion 0 dentro de cmd->data la tomo como signed long *
			tmp16 = (cmd->data);
			// Le asigno el valor de la velocidad ajustada a 1 segundo
			counts_to_stop = (*tmp16);
			// Habilito el chequeo de cuentas para detener el motor
			counts_check = 1;
		break;
 		case DC_MOTOR_GET_ENCODER_TO_STOP:
			/* Obtener la cantidad de las cuentas restantes que quedan por 
			realizar hasta detenerse.
			:DATO:
			-
			:RESP:
			Numero entero de 16 bits con signo, que representa la cantidad
			de cuentas del encoder restantes para detener el motor.
			*/
			// A la posicion 0 dentro de response->data la tomo como signed long *
			tmp16 = (response.data);
			// Le asigno el valor de la velocidad ajustada a 1 segundo
			(*tmp16) = counts_to_stop;
			// Corrijo el largo del paquete
			response.len += 2;
		break;
 		case DC_MOTOR_DONT_STOP:
			/* Deshace los comandos DC_MOTOR_DONT_STOP y DC_MOTOR_GET_ENCODER_TO_STOP, 
			deshabilita el conteo de cuentas para frenar y sigue en el estado actual.
			:DATO:
			-
			:RESP:
			-
			*/
			counts_check = 0;
		break;
 		case DC_MOTOR_MOTOR_CONSUMPTION:
			/* Numero entero positivo de 16 bits en el rango desde 0x0000 hasta
			0x03FF, que representa el consumo promedio del ultimo segundo.
			:DATO:
			-
			:RESP:
			-
			*/
			// A la posicion 0 dentro de response->data la tomo como signed long *
			tmp16 = (response.data);
			// Le asigno el valor del ultimo consumo del motor
			(*tmp16) = last_consumption;
			// Corrijo el largo del paquete
			response.len += 2;
		break;
 		/*case DC_MOTOR_MOTOR_STRESS_ALARM:
		break;
 		case DC_MOTOR_MOTOR_SHUT_DOWN_ALARM:
		break;*/
 		case DC_MOTOR_GET_DC_SPEED:
			/* Obtiene la velocidad del motor en cuentas del encoder por segundo
			:DATO:
			0x00 para sentido horario o 0x01 para sentido anti-horario. Numero
			entero de 16 bits con signo, que representa la velocidad en cuentas
			por segundos.
			:RESP:
			-
			*/
			// Sentido de giro del motor
			if (turn == CLOCKWISE)
			{
				(response.data)[0] = 0x00;
			} else {
				(response.data)[0] = 0x01;
			}
			// A la posicion 1 dentro de response->data la tomo como signed long *
			tmp16 = (response.data) + 1;
			// Le asigno el valor de la velocidad ajustada a 1 segundo
			(*tmp16) = counts_real * INTERVAL_CORRECTION;
			// Corrijo el largo del paquete
			response.len += 2;
		break;

		default:
			response.len++;
			response.cmd = COMMON_ERROR;
			response.data[0] = 0x01; // Comando desconocido
		break;
	}	

	// CRC de la respuesta
	response.crc = generate_8bit_crc((char *)(&response), response.len, CRC_PATTERN);

	return;
}
\end{verbatimtab}
}

\subsection{Placa controladora de sensores}
\label{hA_source_sense}

C\'odigo fuente de la placa controladora de sensores.

{\scriptsize
\begin{verbatimtab}
//CCS PCM V4.023 COMPILER

#define CARD_GROUP	DISTANCE_SENSOR	// Ver protocol.h
#define CARD_ID		1		// Valor entre 0 y E

// ID:0 -> Placa de telemetros
// ID:1 -> Placa de sensores de piso y sensor de ultrasonido
// ID:2 -> Placa de telemetros

// Descripcion de la placa
#define DESC		"PLACA SENSORES 1.0" // Maximo DATA_SIZE bytes

// Determina el tipo de sensores principales en la placa - CAMBIARLO SEGUN CORRESPONDA
// Posibles valores: FLOOR_SENSORS o TELEMETERS_SENSORS o NONE
//#define SENSORS_TYPE	TELEMETERS_SENSORS

// Determina contra que esta conectado el pin TRIGGER - CAMBIARLO SEGUN CORRESPONDA
// Posibles valores: NONE o ULTRASONIC_SENSOR o SWITCH_SENSOR o LED
//#define TRIGGER_TYPE	ULTRASONIC_SENSOR
//#define TRIGGER_TYPE	SWITCH_SENSOR

#if CARD_ID == 0
	#define SENSORS_TYPE	TELEMETERS_SENSORS
	#define TRIGGER_TYPE	SWITCH_SENSOR
	#define DEFAULT_MASK	0x2F
#elif CARD_ID == 1
	#define SENSORS_TYPE	FLOOR_SENSORS
	#define TRIGGER_TYPE	ULTRASONIC_SENSOR
	#define DEFAULT_MASK	0x37
#elif CARD_ID == 2
	#define SENSORS_TYPE	TELEMETERS_SENSORS
	#define TRIGGER_TYPE	SWITCH_SENSOR
	#define DEFAULT_MASK	0x2F
#endif

// Define el estado logico para prender o apagar los sensores - CAMBIARLO SEGUN CORRESPONDA
#define SENSOR_ON	0
#define SENSOR_OFF	1

#define SAMPLES_DEFAULT 5

// Ancho del pulso que se debe enviar al sensor de ultrasonido como INIT - CAMBIARLO SEGUN CORRESPONDA
#define ULTRASONIC_INIT_PULSE_WIDTH_US	15

// Frame de muestreo. Cada este tiempo se realiza una nueva lectura ver CalculoTMR.xls
#define COMPLEMENT_TIME					12411 // 85ms
// Tiempo necesario para realizar la lectura en los telemetros - CAMBIARLO SEGUN CORRESPONDA
#define TELEMETERS_WAITING_TIME_CYCLES	26792 //62ms
#define TELEMETERS_COMPLEMENT_TIME		51161 //23ms complemento a COMPLEMENT_TIME
// Tiempo necesario para realizar la lectura en los sensores de piso - CAMBIARLO SEGUN CORRESPONDA
#define FLOOR_WAITING_TIME_CYCLES		62411 //5ms
#define FLOOR_COMPLEMENT_TIME			15536 //80ms complemento a COMPLEMENT_TIME
// Tiempo necesario para realizar la lectura del sensor de ultrasonido - CAMBIARLO SEGUN CORRESPONDA
#define ULTRASONIC_WAITING_TIME_CYCLES	45536 //32ms
#define ULTRASONIC_COMPLEMENT_TIME		32411 //53ms complemento a COMPLEMENT_TIME

// Tiempo entre el cambio de canal del ADC y una muestra estable (minimo 10us) - CAMBIARLO SEGUN CORRESPONDA
#define ADC_DELAY	20

/* Modulo Generico - main.c
 * PIC16F88 - MAX232 - SENSORES
 *
 *                               PIC16F88
 *                .------------------------------------.
 *       SENSE_3 -|RA2/AN2/CVREF/VREF           RA1/AN1|- SENSE_2
 *       SENSE_4 -|RA3/AN3/VREF+/C1OUT          RA0/AN0|- SENSE_1
 *       SENSE_5 -|RA4/AN4/T0CKI/C2OUT    RA7/OSC1/CLKI|- XT CLOCK pin1, 27pF to GND
 * RST/ICD2:MCLR -|RA5/MCLR/VPP           RA6/OSC2/CLKO|- XT CLOCK pin2, 27pF to GND
 *           GND -|VSS                              VDD|- +5v
 *       TRIGGER -|RB0/INT/CCP1       RB7/AN6/PGD/T1OSI|- ICD2:PGD/SENSOR_5
 *      SENSOR_1 -|RB1/SDI/SDA  RB6/AN5/PGC/T1OSO/T1CKI|- ICD2:PGC/SENSOR_4
 *  MAX232:R1OUT -|RB2/SDO/RX/DT           RB5/SS/TX/CK|- MAX232:T1IN
 *     	SENSOR_2 -|RB3/PGM/CCP1             RB4/SCK/SCL|- SENSOR_3
 *                '------------------------------------'
 *    
 */

#include <16F88.h>
#DEVICE ADC = 10
#include <stdio.h>
#include <string.h>
#include <stdlib.h>

#fuses HS,NOWDT,NOPROTECT,NOLVP
#use delay (clock=20000000)

#use rs232(BAUD=115200,PARITY=N,XMIT=PIN_B5,RCV=PIN_B2,BITS=8,ERRORS,TIMEOUT=1,STOP=1,UART1)
#use fast_io(A)
#use fast_io(B)

#byte porta=0x05
#byte portb=0x06

// MAX232
#bit tx=portb.5
#bit rx=portb.2

// Pin para el control del sensore de ultrasonido, switch on/off o led
#bit trigger=portb.0
// Led
#bit led1=portb.0
// Pin para switch On/Off
#bit inOnOff=portb.0
// Pin de control del sensor de ultrasonido
#bit ultrasonido=portb.0

// Sensores - Base de los transistores
#bit sensor1=portb.1
#bit sensor2=portb.3
#bit sensor3=portb.4
#bit sensor4=portb.6
#bit sensor5=portb.7

// Sensores - Salida de los sensores (SENSE)
#bit sense1=porta.0
#bit sense2=porta.1
#bit sense3=porta.2
#bit sense4=porta.3
#bit sense5=porta.4

// Estados para la maquina de estados que controla las funciones de la placa
#define STATE_FREE			0
#define STATE_START_READING	1 
#define STATE_WAIT_TO_READ	2
#define STATE_READ_VALUES	3
#define STATE_WAITING		4

// Estados para la lectura del sensor de ultrasonido
#define USONIC_STATE_START	0
#define USONIC_STATE_STOP	1 

// Posibles conexiones del pin TRIGGER
#define NONE			  (0x00)
#define ULTRASONIC_SENSOR (NONE + 1)
#define SWITCH_SENSOR     (ULTRASONIC_SENSOR + 1)
#define LED				  (SWITCH_SENSOR + 1)

// Posibles sensores principales
#define TELEMETERS_SENSORS (NONE + 1)
#define FLOOR_SENSORS      (TELEMETERS_SENSORS + 1)

// Tiempo maximo para NONE
#define NONE_TIME	65530 //0ms
#define SWITCH_TIME 65530 //0ms
#define LED_TIME 	65530 //0ms

#if SENSORS_TYPE == FLOOR_SENSORS
	#if TRIGGER_TYPE == ULTRASONIC_SENSOR
		#define SENSORS_WAITING_TIME ULTRASONIC_WAITING_TIME_CYCLES
		#define WAITING_TIME ULTRASONIC_COMPLEMENT_TIME //Complemento 85ms -> 53ms
	#else
		#define SENSORS_WAITING_TIME FLOOR_WAITING_TIME_CYCLES
		#define WAITING_TIME FLOOR_COMPLEMENT_TIME //Complemento 85ms -> 80ms
	#endif
#elif SENSORS_TYPE == TELEMETERS_SENSORS
	#define SENSORS_WAITING_TIME TELEMETERS_WAITING_TIME_CYCLES
	#define WAITING_TIME TELEMETERS_COMPLEMENT_TIME //Complemento 85ms -> 23ms
#elif SENSORS_TYPE == NONE
	#if TRIGGER_TYPE == ULTRASONIC_SENSOR
		#define SENSORS_WAITING_TIME ULTRASONIC_WAITING_TIME_CYCLES
		#define WAITING_TIME ULTRASONIC_COMPLEMENT_TIME //Complemento 85ms -> 53ms
	#else
		#define SENSORS_WAITING_TIME NONE_TIME
		#define WAITING_TIME COMPLEMENT_TIME //85ms
	#endif
#endif

#include <../../protocolo/src/protocol.c>
/*
** Variables definidas en protocol.c

short reset; // Variable para hacer el reset
short crcOK; // Informa si el CRC del paquete parseado fue correcto
short sendResponse; // Informa que no debe mandarse la respuesta automatica

char buffer[MAX_BUFFER_SIZE];	// Buffer de recepcion de comandos
int buffer_write;				// Indice de escritura
int buffer_read;				// Indice de lectura
int data_length;				// Largo de los datos en el buffer

struct command_t command; 	// Comando parseado
struct command_t response; 	// Respuesta

** Implementar las siguientes funciones (usadas por el protocolo)

void init(); // Inicializa puertos y variables
void doCommand(struct command_t * cmd); // Examina y ejecula el comando

***/

// IO como entrada o salida
int trisB_value = 0b00100101;

// Determina el estado actual
int state;

// Vector donde se almacenan los valores de los sensores
unsigned long values[6];

// Cantidad de muestras a tomar por sensor
int samples;

// Mascara que ignora ciertos sensores
int sensorMask;

// Determina cuales sensores estan siendo leidos en este instante
int actualReadSensor;

// Determina cuales son los sensores de los que se pide la lectura
int readSensor;

// Bufferea el pedido de sensores de los que se pide la lectura
int bufferedReadSensor;

// Determina el destinatario actual
int actalTO;

// Informa quien hizo el pedido
int requestFrom;

// Informa de quien vino el ultimo pedido (mientras habia una lectura en curso)
int bufferedFrom;

// Determina el comando actual
int actalCmd;

// Informa cual fue el comando del pedido
int requestCmd;

// Informa cual fue el comando del ultimo pedido (mientras habia una lectura en curso)
int bufferedCmd;

// Estado para el sensor de ultrasonido
short usonic_state;

// Tiempo para el comienzo del pulso del sensor de ultrasonido
unsigned long pulseStart;

// Flag de aviso sobre la interrupcion en TIMER1
short intTMR;

// Tipo de alarma en trigger
int alarmType;

// Flag de alarma en trigger
short triggerAlarm;

/* Habilita los sensores segun corresponda para comenzar la lectura */
void startReading(int sensors);

/* Realiza la lectura sobre los sensores segun corresponda */
void readSensors(int sensors);

/* Interrupcion del TIMER1 */
void int_timer(void);

/* Interrupcion de RB0 */
void int_trigger(void);

/* Crea la respuesta sobre la lectura de los sensores */
void sendValues(int to, int cmd, long * values, int sensors);

/* Interrupcion del TIMER1 */
#INT_TIMER1
void int_timer(void)
{
	// Habilita el flag de aviso que sucedio la interrupcion
	intTMR = 1;
	disable_interrupts(INT_TIMER1);
}

/* Interrupcion de RB0 */
#INT_EXT
void int_trigger(void)
{
#if TRIGGER_TYPE == ULTRASONIC_SENSOR
	if (usonic_state == USONIC_STATE_START)
	{
		// Tomo el tiempo en que comienza el pulso
		pulseStart = get_timer1();
		// Cambio el tipo de flanco
		ext_int_edge(H_TO_L);
		// Cambio el estado
		usonic_state = USONIC_STATE_STOP;
	} else {
		// Tomo el tiempo y guardo el valor
		values[5] = (get_timer1() - pulseStart)/2;
		// Deshabilita la interupcion
		disable_interrupts(INT_EXT);
	}
#elif TRIGGER_TYPE == SWITCH_SENSOR
	triggerAlarm = 1;
#endif


}

void init()
{
	// Inicializa puertos
	set_tris_a(0b11111111);
	set_tris_b(trisB_value);

	// ***ADC***
	setup_port_a(sAN0);
	setup_adc(ADC_CLOCK_INTERNAL);
	set_adc_channel(0);
	setup_adc_ports(sAN0);
	setup_vref(VREF_HIGH | 8);
	
	// Seteo el Timer1 como fuente interna
	setup_timer_1(T1_INTERNAL | T1_DIV_BY_8);
	set_timer1(26786);
	
	// Interrupcion sobre el Timer1
	enable_interrupts(INT_TIMER1);

	// Seteo el pin RB0 - Sensor de ultrasonido :)
	ext_int_edge(L_TO_H);
	enable_interrupts(INT_EXT);

	// Habilito las interrupciones
	enable_interrupts(GLOBAL);
	
	// Variable para hacer el reset
	reset = false;

	// Apaga todos los sensores
	sensor1 = SENSOR_OFF;
	sensor2 = SENSOR_OFF;
	sensor3 = SENSOR_OFF;
	sensor4 = SENSOR_OFF;
	sensor5 = SENSOR_OFF;

	// Inicializa los valores
	values[0] = 0x0000;
	values[1] = 0x0000;
	values[2] = 0x0000;
	values[3] = 0x0000;
	values[4] = 0x0000;
	values[5] = 0x0000;

	// Muestras iniciales
	samples = SAMPLES_DEFAULT;
	
	// Inicializa la mascara -> todos habilitados (0x3F)
	sensorMask = DEFAULT_MASK;
	
	//Determina el estado actual
	state = STATE_FREE;
	
	// Sin lectura temprana
	readSensor = 0x00;
	bufferedReadSensor = 0x00;
	actalTO = 0x00;
	requestFrom = 0x00;
	bufferedFrom = 0x00;
	actalCmd = 0x00;
	requestCmd = 0x00;
	bufferedCmd = 0x00;

	alarmType = 0x00;
	triggerAlarm = 0;
#if TRIGGER_TYPE == SWITCH_SENSOR
	disable_interrupts(INT_EXT);
#endif

#if TRIGGER_TYPE == LED
	// TRIGGER como escritura
	bit_clear(trisB_value, 0);
	set_tris_b(trisB_value);
#endif

	return;	
}	

void sendAlarm()
{
	/*TODO: revisar protocolo*/	
	triggerAlarm = 0;
	return;
}	

void main()
{
	// Placa Generica - Implementacion del protocolo
	init();

	// Init del protocol
	initProtocol();

	// FOREVER
	while(true)
	{

		// Ejecucion de la maquina de estados
		switch (state)
		{
			case STATE_FREE:
				/*
				* Analiza los paquetes y retransmite o lanza el pedido de lectura
				*/
				
#if TRIGGER_TYPE == SWITCH_SENSOR
				// Enviar alarma de trigger
				if (triggerAlarm == 1)
					sendAlarm();
#endif

				// Protocolo
				runProtocol(&command);

				if (readSensor != 0x00)
				{
					// Almacena el pedido sobre los sensores
					actualReadSensor = readSensor;
					readSensor = 0x00;
					actalTO = requestFrom;
					actalCmd = requestCmd;
					// Cambio de estado
					state = STATE_START_READING;
				}	
				break;
			case STATE_START_READING:
				/*
				* Genera un pedido de lectura a los sensores que lo necesiten
				* y setea los TIMERs para contabilizar los tiempos
				*/
				
				// Limpio la variable
				bufferedReadSensor = 0x00;

				// Inhabilita los sensores enmascarados
				actualReadSensor &= sensorMask;
				
				// Pedido de -START-
				startReading(actualReadSensor);
				
				// Flag de interrupcion
				intTMR = 0;
				
				// Seteo de TIMERs
				set_timer1(SENSORS_WAITING_TIME);
				enable_interrupts(INT_TIMER1);
				
				// Cambio de estado
				state = STATE_WAIT_TO_READ;
				
				break;
			case STATE_WAIT_TO_READ:
				/*
				* Espera el tiempo necesario para tomar las muestras en los sensores
				* Recibe nuevos pedidos y los agraga para una proxima lectura
				*/

#if TRIGGER_TYPE == SWITCH_SENSOR
				// Enviar alarma de trigger
				if (triggerAlarm == 1)
					sendAlarm();
#endif

				// Protocolo
				runProtocol(&command);
				
				// Almacena el pedido sobre los sensores si no hay otro ya
				if ((readSensor != 0x00) && (bufferedReadSensor == 0x00))
				{
					bufferedReadSensor = readSensor;
					bufferedFrom = requestFrom;
					bufferedCmd = requestCmd;
					readSensor = 0x00;
				}
				
				// Cambio de estado?
				if (intTMR == 1)
				{
					// El TIMER1 hizo timeout -> leer sensores
					state = STATE_READ_VALUES;
				}	
				
				break;
			case STATE_READ_VALUES:
				/*
				* Toma las muestras en los sensores y envia el paquete de respuesta
				*/
				
				// Toma las muestras de los sensores
				readSensors(actualReadSensor);
				
				// Mandar paquete de respuesta
				sendValues(actalTO, actalCMD, values, actualReadSensor);
				
				// Flag de interrupcion
				intTMR = 0;
				
				// Setea el tiempo de espera y pasa al estado de espera
				set_timer1(WAITING_TIME);
				enable_interrupts(INT_TIMER1);
				
				state = STATE_WAITING;
				
				break;
			case STATE_WAITING:
				/*
				* Espera a que pase el tiempo de espera entre lecturas para evitar rebotes
				* Recibe nuevos pedidos y los agraga para una proxima lectura
				*/
				
#if TRIGGER_TYPE == SWITCH_SENSOR
				// Enviar alarma de trigger
				if (triggerAlarm == 1)
					sendAlarm();
#endif

				// Protocolo
				runProtocol(&command);
				
				// Almacena el pedido sobre los sensores si no hay otro ya
				if ((readSensor != 0x00) && (bufferedReadSensor == 0x00))
				{
					bufferedReadSensor = readSensor;
					bufferedFrom = requestFrom;
					bufferedCmd = requestCmd;
					readSensor = 0x00;
				}
				
				// Cambio de estado?
				if (intTMR == 1)
				{
					// El TIMER1 hizo timeout
					if (bufferedReadSensor != 0x00)
					{
						// Hay un pedido pendiente y lo carga
						actualReadSensor = bufferedReadSensor;
						bufferedReadSensor = 0x00;
						actalTO = bufferedFrom;
						actalCmd = bufferedCmd;
						// Cambio de estado 
						state = STATE_START_READING;
					} else {
						state = STATE_FREE;
					}	
				}

				break;
			default:
				init();
				break;
		}
	}

	return;
}

/* Habilita los sensores segun corresponda para comenzar la lectura*/
void startReading(int sensors)
{
	// Sensor1
	if (bit_test(sensors, 0) == 1)
		sensor1 = SENSOR_ON;

	// Sensor2
	if (bit_test(sensors, 1) == 1)
		sensor2 = SENSOR_ON;

	// Sensor3
	if (bit_test(sensors, 2) == 1)
		sensor3 = SENSOR_ON;

	// Sensor4
	if (bit_test(sensors, 3) == 1)
		sensor4 = SENSOR_ON;

	// Sensor5
	if (bit_test(sensors, 4) == 1)
		sensor5 = SENSOR_ON;

	// Sensor6
#if TRIGGER_TYPE == ULTRASONIC_SENSOR
	// Sensor6 -> ULTRASONIC_SENSOR
	if (bit_test(sensors, 5) == 1)
	{
		// Comienza el pulso de habilitacion -> TRIGGER como escritura
		bit_clear(trisB_value, 0);
		set_tris_b(trisB_value);
		// Pin en estado habilitado -> envio del pulso INIT
		trigger = 1;
		delay_us(ULTRASONIC_INIT_PULSE_WIDTH_US);
		trigger = 0;
		// Termina el pulso de habilitacion -> TRIGGER como lectura
		bit_set(trisB_value, 0);
		set_tris_b(trisB_value);
		// Setea la interrupcion sobre RB0 en flanco ascendente
		ext_int_edge(L_TO_H);
		// Seteo el estado actual del pulso del sensor de ultrasonido
		usonic_state = USONIC_STATE_START;
		// Habilita la interrupcion
		enable_interrupts(INT_EXT);
	}
#elif TRIGGER_TYPE == SWITCH_SENSOR
	// Sensor6 -> SWITCH_SENSOR
	if (bit_test(sensors, 5) == 1)
	{
		if (trigger == 1)
			values[5] = 0xFFFF;
		else
			values[5] = 0;		
	} 
#endif

	return;
}	

/* Realiza la lectura sobre los sensores segun corresponda */
void readSensors(int sensors)
{
	int i;
	
	values[0] = 0;
	values[1] = 0;
	values[2] = 0;
	values[3] = 0;
	values[4] = 0;
	
	for (i = 0; i < samples; i++)
	{	
		// Sensor1
		if (bit_test(sensors, 0) == 1)
		{
			// ADC en el pin correcto
			set_adc_channel(0);
			// Espera el tiempo necesario
			delay_us(ADC_DELAY);
			// Toma la muestra
			values[0] += read_adc();
		}
		
		// Sensor2
		if (bit_test(sensors, 1) == 1)
		{
			// ADC en el pin correcto
			set_adc_channel(1);
			// Espera el tiempo necesario
			delay_us(ADC_DELAY);
			// Toma la muestra
			values[1] += read_adc();
		}
		
		// Sensor3
		if (bit_test(sensors, 2) == 1)
		{
			// ADC en el pin correcto
			set_adc_channel(2);
			// Espera el tiempo necesario
			delay_us(ADC_DELAY);
			// Toma la muestra
			values[2] += read_adc();
		}
		
		// Sensor4
		if (bit_test(sensors, 3) == 1)
		{
			// ADC en el pin correcto
			set_adc_channel(3);
			// Espera el tiempo necesario
			delay_us(ADC_DELAY);
			// Toma la muestra
			values[3] += read_adc();
		}
		
		// Sensor5
		if (bit_test(sensors, 4) == 1)
		{
			// ADC en el pin correcto
			set_adc_channel(4);
			// Espera el tiempo necesario
			delay_us(ADC_DELAY);
			// Toma la muestra
			values[4] += read_adc();
		}
	}
	
	/*sensor1 = SENSOR_OFF;
	sensor2 = SENSOR_OFF;
	sensor3 = SENSOR_OFF;
	sensor4 = SENSOR_OFF;
	sensor5 = SENSOR_OFF;*/
	
	values[0] /= samples;
	values[1] /= samples;
	values[2] /= samples;
	values[3] /= samples;
	values[4] /= samples;
	
	return;
}	

/* Crea la respuesta sobre la lectura de los sensores */
void sendValues(int to, int cmd, long * values, int sensors)
{
	int idx = 1, i;
	signed long * tmp16;
	
	command.len = MIN_LENGTH + 1;
	command.to = to;
	command.from = THIS_CARD;
	command.cmd = cmd;
	command.data[0] = sensors;

	// Valores de los sensores en command.data segun corresponda
	for (i = 0; i < 6; i++)
	{
		if (bit_test(sensors, i) == 1)
		{
			// A la posicion 0 dentro de command.data la tomo como signed long *
			tmp16 = (command.data + idx);
			// Le asigno el valor del sensor
			(*tmp16) = values[i];
			idx+=2;
			command.len += 2;
		}
	}

	command.crc = generate_8bit_crc((char *)(&command), command.len, CRC_PATTERN);
	// Envio del comando
	send(&command);

	return;
}

/* Verifica que el comando sea valido y lo ejecuta */
void doCommand(struct command_t * cmd)
{
	int crc, i, len;
		
	// Calculo del CRC
	crc = generate_8bit_crc((char *)cmd, cmd->len, CRC_PATTERN);
	
	// CRC ok?
	if (cmd->crc != crc)
	{		
		// Creo respuesta de error
		response.len = MIN_LENGTH + cmd->len + 2 + 1;
		response.to = cmd->from;
		response.from = THIS_CARD;
		response.cmd = COMMON_ERROR;
		response.data[0] = 0x00;
		// Agrego el paquete que contiene el error de CRC
		response.data[1] = cmd->len;
		response.data[2] = cmd->to;
		response.data[3] = cmd->from;
		response.data[4] = cmd->cmd;
		// Campo data
		len = cmd->len - MIN_LENGTH;
		for (i = 0; i < len; i++)
			response.data[5 + i] = (cmd->data)[i];
		// CRC erroneo
		response.data[5 + len] = cmd->crc;
		// CRC esperado
		response.data[5 + len + 1] = crc;
		// CRC de la respuesta
		response.crc = generate_8bit_crc((char *)(&response), response.len, CRC_PATTERN);
	
		crcOK = false;
		return;
	}

	crcOK = true;
	
	// Minimo todos setean esto
	response.len = MIN_LENGTH;
	response.to = cmd->from & 0x77;
	response.from = THIS_CARD;
	response.cmd = cmd->cmd | 0x80;

	switch (cmd->cmd)
	{
		// Comandos comunes
		case COMMON_INIT:
			init();
			// Enviar la descripcion de la placa en texto plano
			strcpy(response.data, DESC);
			response.len += strlen(response.data);
		break;
		case COMMON_RESET:
			// Enviar la descripcion de la placa en texto plano
			strcpy(response.data, DESC);
			response.len += strlen(response.data);
			// Reset!
			reset = true;
		break;
		case COMMON_PING:
			// No hace falta hacer mas nada
		break;
 		case COMMON_ERROR:
			// Por ahora se ignora el comando
		break;
		
		/* Comandos especificos */

 		case DISTANCE_SENSOR_ON_DISTANCE_SENSOR:
			/* Enciende el sensor de distancia indicado.
			:DATO:
			Valor de 0x00 a 0x05 que representa el ID del sensor a encender.
			El ID 0x05 hace referencia al led de la placa si esta presente, 
			en caso contrario se ignora.
			:RESP:
			-
			*/
			// Enciende los sensores segun corresponda
			if ((cmd->data)[0] == 0)
				// Sensor1
				sensor1 = SENSOR_ON;
			else if ((cmd->data)[0] == 1)
				// Sensor2
				sensor2 = SENSOR_ON;
			else if ((cmd->data)[0] == 2)
				// Sensor3
				sensor3 = SENSOR_ON;
			else if ((cmd->data)[0] == 3)
				// Sensor4
				sensor4 = SENSOR_ON;
			else if ((cmd->data)[0] == 4)
				// Sensor5
				sensor5 = SENSOR_ON;
#if TRIGGER_TYPE == LED
			else if ((cmd->data)[0] == 5)
				// Led
				led1 = 1;
#endif
		break;
 		case DISTANCE_SENSOR_OFF_DISTANCE_SENSOR:
			/* Apaga el sensor de distancia indicado.
			:DATO:
			Valor de 0x00 a 0x05 que representa el ID del sensor a apagar. El
			ID 0x05 hace referencia al sensor de ultrasonido o switch de la placa.
			:RESP:
			-
			*/
			// Apaga los sensores segun corresponda
			if ((cmd->data)[0] == 0)
				// Sensor1
				sensor1 = SENSOR_OFF;
			else if ((cmd->data)[0] == 1)
				// Sensor2
				sensor2 = SENSOR_OFF;
			else if ((cmd->data)[0] == 2)
				// Sensor3
				sensor3 = SENSOR_OFF;
			else if ((cmd->data)[0] == 3)
				// Sensor4
				sensor4 = SENSOR_OFF;
			else if ((cmd->data)[0] == 4)
				// Sensor5
				sensor5 = SENSOR_OFF;
#if TRIGGER_TYPE == LED
			else if ((cmd->data)[0] == 5)
				// Led
				led1 = 0;
#endif
		break;
 		case DISTANCE_SENSOR_SET_MASK:
			/* Habilita o deshabilita cada uno de los sensores de distancia conectados al
			controlador. Permite identificar los sensores a los que se debera tener en
			cuenta para futuras lecturas.
			:DATO:
			Valor de 0x00 a 0x3F donde cada bit representa el ID del sensor a
			habilitar o deshabilitar. Si 2^ID = 1 entonces el sensor ID esta habilitado.
			Si 2^ID = 0 entonces el sensor ID esta deshabilitado.
			:RESP:
			-
			*/
			sensorMask = (cmd->data)[0];
		break;
 		case DISTANCE_SENSOR_GET_MASK:
			/* Obtiene el estado de habilitacion de cada uno de los sensores de distancia
			conectados al controlador.
			:DATO:
			-
			:RESP:
			Valor de 0x00 a 0x3F donde cada bit representa el ID del sensor a
			habilitar o deshabilitar. Si 2^ID = 1 entonces el sensor ID esta habilitado.
			Si 2^ID = 0 entonces el sensor ID esta deshabilitado.
			*/
			// Envio la mascara de sensores
			response.data[0] = sensorMask;
			// Corrijo el largo del paquete
			response.len++;
		break;
 		case DISTANCE_SENSOR_GET_VALUE:
			/* Obtiene el valor promedio de la entrada de los sensores indicados.
			:DATO:
			Valor de 0x00 a 0x3F donde cada bit representa el ID del sensor
			del cual obtener la lectura.
			:RESP:
			Valor de 0x00 a 0x3F donde cada bit representa el ID del sensor
			del cual proviene el la lectura de distancia. Secuencia de numeros enteros
			positivos de 16 bits en el rango desde 0x0000 hasta 0x03FF, con el valor de
			la lectura que representa la distancia al objeto. En la secuencia de numeros
			el orden esta dado de izquierda a derecha comenzando por el bit menos
			significativo.
			En el caso del sensor de ultrasonido el rango es desde 0x0000 hasta 0x7594
			que representa la minima y maxima lectura del sensor.
			En el caso del switch, un estado logico bajo se lee como 0x0000 y un estado
			logico alto se lee como 0xFFFF.
			*/
			readSensor = (cmd->data)[0];
			requestFrom = 	response.to;
			requestCmd = response.cmd;
			sendResponse = false;
			samples = SAMPLES_DEFAULT;
		break;
 		case DISTANCE_SENSOR_GET_ONE_VALUE:
			/* Obtiene el valor de la entrada del sensor indicado. Igual al comando 8.5 pero
			sin realizar un promedio de lecturas.
			:DATO:
			Valor de 0x00 a 0x3F donde cada bit representa el ID del sensor
			del cual obtener la lectura.
			:RESP:
			Valor de 0x00 a 0x3F donde cada bit representa el ID del sensor
			del cual proviene el la lectura de distancia. Secuencia de numeros enteros
			positivos de 16 bits en el rango desde 0x0000 hasta 0x03FF, con el valor de
			la lectura que representa la distancia al objeto. En la secuencia de numeros
			el orden esta dado de izquierda a derecha comenzando por el bit menos
			significativo.
			En el caso del sensor de ultrasonido el rango es desde 0x0000 hasta 0x7594
			que representa la minima y maxima lectura del sensor.
			En el caso del switch, un estado logico bajo se lee como 0x0000 y un estado
			logico alto se lee como 0xFFFF.
			*/
			readSensor = (cmd->data)[0];
			requestFrom = 	response.to;
			requestCmd = response.cmd;
			sendResponse = false;
			samples = 1;
		break;
		case DISTANCE_SENSOR_ALARM_ON_STATE:
			/* Cuando un switch esta presente en el ID: 0x05, establece si se desea o no
			recibir una alarma ante cierto cambio de estado en el mismo. Puede ser ante
			cualquier cambio o sobre un flanco ascendente o descendente.
			:DATO:
			Valor entre 0x00 y 0x03 con el tipo de cambio ante el cual generar
			la alarma. Con un 0x00 ignora cualquier cambio en el switch. Se utiliza
			0x01 para que cualquier cambio en el switch genere el mensaje, 0x02 para
			que sea solo ante un flanco ascendente y 0x03 para que sea solo ante un
			flanco descendente.
			:RESP:
			-
			*/
#if TRIGGER_TYPE == SWITCH_SENSOR
			switch ((cmd->data)[0])
			{
				case 0x00:
					// Ignorar
					disable_interrupts(INT_EXT);
					break;
				case 0x01:
					// Cualquier cambio
					enable_interrupts(INT_EXT);
					break;
				case 0x02:
					// flanco ascendente
					ext_int_edge(H_TO_L);
					enable_interrupts(INT_EXT);
					break;
				case 0x03:
					// flanco descendente
					ext_int_edge(L_TO_H);
					enable_interrupts(INT_EXT);
					break;
				default:
					(cmd->data)[0] = 0x00;
					break;
			}
			alarmType = (cmd->data)[0];
#endif
		break;
		default:
			response.len++;
			response.cmd = COMMON_ERROR;
			response.data[0] = 0x01; // Comando desconocido
		break;
	}	

	// CRC de la respuesta
	response.crc = generate_8bit_crc((char *)(&response), response.len, CRC_PATTERN);

	return;
}
\end{verbatimtab}
}


\subsection{Placa controladora de servo motores}
\label{hA_source_servo}

C\'odigo fuente de la placa controladora de servo motores.

{\scriptsize
\begin{verbatimtab}
//CCS PCM V4.023 COMPILER

#define CARD_GROUP	SERVO_MOTOR	// Ver protocol.h
#define CARD_ID		0			// Valor entre 0 y E

// Descripcion de la placa
#define DESC		"SERVOR CONTROL - 1.0" // Maximo DATA_SIZE bytes

/* Modulo Servo - main.c
 * PIC16F88 - MAX232 - SERVO
 *
 *                               PIC16F88
 *                .------------------------------------.
 *       MOTOR_4 -|RA2/AN2/CVREF/VREF           RA1/AN1|- MOTOR_5
 *           LED -|RA3/AN3/VREF+/C1OUT          RA0/AN0|- 
 *           LED -|RA4/AN4/T0CKI/C2OUT    RA7/OSC1/CLKI|- XT CLOCK pin1, 27pF to GND
 * RST/ICD2:MCLR -|RA5/MCLR/VPP           RA6/OSC2/CLKO|- XT CLOCK pin2, 27pF to GND
 *           GND -|VSS                              VDD|- +5v
 *       MOTOR_1 -|RB0/INT/CCP1       RB7/AN6/PGD/T1OSI|- ICD2:PGD
 *       MOTOR_2 -|RB1/SDI/SDA  RB6/AN5/PGC/T1OSO/T1CKI|- ICD2:PGC
 *  MAX232:R1OUT -|RB2/SDO/RX/DT           RB5/SS/TX/CK|- MAX232:T1IN
 *     ICD2:PGM/ -|RB3/PGM/CCP1             RB4/SCK/SCL|- 
 *     MOTOR_3    '------------------------------------'
 *    
 */

#include <16F88.h>
#DEVICE ADC = 10
#include <stdio.h>
#include <string.h>
#include <stdlib.h>

#fuses HS,NOWDT,NOPROTECT,NOLVP
#use delay (clock=20000000)

#use rs232(BAUD=115200,PARITY=N,XMIT=PIN_B5,RCV=PIN_B2,BITS=8,ERRORS,TIMEOUT=1,STOP=1,UART1)
#use fast_io(A)
#use fast_io(B)

#byte porta=0x05
#byte portb=0x06

// Led
#bit led1=porta.1
#bit led2=porta.4

// MAX232
#bit tx=portb.5
#bit rx=portb.2

// PWMs
#bit pwm1=portb.0
#bit pwm2=portb.1
#bit pwm3=portb.3
#bit pwm4=porta.2
#bit pwm5=porta.1

#include <../../protocolo/src/protocol.c>
/*
** Variables definidas en protocol.c

short reset; // Variable para hacer el reset
short crcOK; // Informa si el CRC del paquete parseado fue correcto
short sendResponse; // Informa que no debe mandarse la respuesta automatica

char buffer[MAX_BUFFER_SIZE];	// Buffer de recepcion de comandos
int buffer_write;				// Indice de escritura
int buffer_read;				// Indice de lectura
int data_length;				// Largo de los datos en el buffer

struct command_t command; 	// Comando parseado
struct command_t response; 	// Respuesta

** Implementar las siguientes funciones (usadas por el protocolo)

void init(); // Inicializa puertos y variables
void doCommand(struct command_t * cmd); // Examina y ejecuta el comando

***/

// Software PWM - Minimo 1750 (~0.71ms)
#define PULSE_MIN	1750
// Tiempo maximo que puede durar un pulso - Maximo 6755(~2.71ms)
#define PULSE_MAX	6755
// Tiempo entre pulsos (~25ms -> ~22.29ms fijos de espera)
#define PWM_MAX		62500
// 1� ~ 27.8 cuentas ~ 69.4us
#define DEGREE		27.8f

// Valor que representa el ancho del pulso para cada servo
long pwm_t[5];
// Angulo de cada servo
long pos[5];
// On/Off de cada servo
short servo[5];

void init()
{
	// Inicializa puertos
	set_tris_a(0b11100001);
	set_tris_b(0b11110100);

	// Seteo el Timer1 como fuente interna
	setup_timer_1(T1_INTERNAL | T1_DIV_BY_2);
	set_timer1(0);

	// Variable para hacer el reset
	reset = false;

	// Valor que representa el ancho del pulso para cada servo
	pwm_t[0] = PULSE_MIN;
	pwm_t[1] = PULSE_MIN;
	pwm_t[2] = PULSE_MIN;
	pwm_t[3] = PULSE_MIN;
	pwm_t[4] = PULSE_MIN;
	
	// Angulo de cada servo
	pos[0] = 0;
	pos[1] = 0;
	pos[2] = 0;
	pos[3] = 0;
	pos[4] = 0;
	
	// On/Off de cada servo
	servo[0] = 0;
	servo[1] = 0;
	servo[2] = 0;
	servo[3] = 0;
	servo[4] = 0;
	
	// Activo los servos segun este o no habilitado
	pwm1 = servo[0];
	pwm2 = servo[1];
	pwm3 = servo[2];
	pwm4 = servo[3];
	pwm5 = servo[4];
	
	return;	
}	

void main()
{
	short check_comm = 0;
	long tmr1;
	
	// Control de servomotores
	init();

	// Init del protocol
	initProtocol();


	// FOREVER
	while(true)
	{
		// Software PWM

		// Tomo el tiempo
		tmr1 = get_timer1();
		
		// Es hora de reiniciar el pulso?
		if (tmr1 >= PWM_MAX)
		{
			// Detiene el analisis de comandos
			check_comm = 0;
			
			// Inicio del periodo
			set_timer1(0);
			// Activo los servos segun este o no habilitado
			pwm1 = servo[0];
			pwm2 = servo[1];
			pwm3 = servo[2];
			pwm4 = servo[3];
			pwm5 = servo[4];
			
		} else
		// Llego al final del pulso?
		if (tmr1 >= PULSE_MAX)
		{
			// Pone las salidas a 0
			pwm1 = 0;
			pwm2 = 0;
			pwm3 = 0;
			pwm4 = 0;
			pwm5 = 0;
			// Analiza si hay comandos para ser atendidos
			check_comm = 1;
			
		} else {
			// Tomo el tiempo
			tmr1 = get_timer1();

			// Es tiempo de desactivar el PWM?
			if (tmr1 >= pwm_t[0])
				pwm1 = 0;
			if (tmr1 >= pwm_t[1])
				pwm2 = 0;
			if (tmr1 >= pwm_t[2])
				pwm3 = 0;
			if (tmr1 >= pwm_t[3])
				pwm4 = 0;
			if (tmr1 >= pwm_t[4])
				pwm5 = 0;
		}
		
		// Protocolo
		if (check_comm == 1)
			runProtocol(&command);
	}

	return;
}

/* Verifica que el comando sea valido y lo ejecuta */
void doCommand(struct command_t * cmd)
{
	int crc, i, len;
		
	// Calculo del CRC
	crc = generate_8bit_crc((char *)cmd, cmd->len, CRC_PATTERN);
	
	// CRC ok?
	if (cmd->crc != crc)
	{		
		// Creo respuesta de error
		response.len = MIN_LENGTH + cmd->len + 2 + 1;
		response.to = cmd->from;
		response.from = THIS_CARD;
		response.cmd = COMMON_ERROR;
		response.data[0] = 0x00;
		// Agrego el paquete que contiene el error de CRC
		response.data[1] = cmd->len;
		response.data[2] = cmd->to;
		response.data[3] = cmd->from;
		response.data[4] = cmd->cmd;
		// Campo data
		len = cmd->len - MIN_LENGTH;
		for (i = 0; i < len; i++)
			response.data[5 + i] = (cmd->data)[i];
		// CRC erroneo
		response.data[5 + len] = cmd->crc;
		// CRC esperado
		response.data[5 + len + 1] = crc;
		// CRC de la respuesta
		response.crc = generate_8bit_crc((char *)(&response), response.len, CRC_PATTERN);
	
		crcOK = false;
		return;
	}

	crcOK = true;
	
	// Minimo todos setean esto
	response.len = MIN_LENGTH;
	response.to = cmd->from & 0x77;
	response.from = THIS_CARD;
	response.cmd = cmd->cmd | 0x80;

	switch (cmd->cmd)
	{
		// Comandos comunes
		case COMMON_INIT: 
			init();
			// Enviar la descripcion de la placa en texto plano
			strcpy(response.data, DESC);
			response.len += strlen(response.data);
		break;
		case COMMON_RESET: 
			// Enviar la descripcion de la placa en texto plano
			strcpy(response.data, DESC);
			response.len += strlen(response.data);
			// Reset!
			reset = true;
		break;
		case COMMON_PING: 
			// No hace falta hacer mas nada
		break;
 		case COMMON_ERROR:
			// Por ahora se ignora el comando
		break;
		
		/* Comandos especificos */
		
 		case SERVO_MOTOR_SET_POSITION:
			/* Determina la posicion en la que debe colocarse el 
			servo motor indicado.
			:DATO:
			Valor de 0x00 a 0x04 que determina el id del servo al
			que se le aplicara la posicion. Valor entre 0x00 y 0xB4
			que representa el rango de 0 a 180 con 1 de presicion.
			:RESP:
			-
			*/
			i = ((cmd->data)[0] & 0x07); // Servo destinatario
			if (i < 5)
			{
				servo[i] = 1;
				pos[i] = (unsigned char)((cmd->data)[1]);
				pwm_t[i] = PULSE_MIN + pos[i] * DEGREE;
			}	
		break;
 		case SERVO_MOTOR_SET_ALL_POSITIONS:
			/* Determina las posiciones en la que deben colocarse
			cada uno de los servomotores
			:DATO:
			Consta de 5 valores entre 0x00 y 0xB4 concatenados, uno
			para cada uno de los servos conectados al controlador.
			Cada valor representa el rango de 0 a 180 con 1 de presicion.
			:RESP:
			-
			*/
			for (i = 0; i < 5; i++)
			{
				servo[i] = 1;
				pos[i] = (unsigned char)((cmd->data)[i]);
				pwm_t[i] = PULSE_MIN + pos[i] * DEGREE;
			}	
		break;
 		case SERVO_MOTOR_GET_POSITION:
			/* Obtiene la ultima posicion del servomotor indicado.
			:DATO:
			Valor de 0x00 a 0x04 que determina el id del servo del que
			se requiere la posicion.
			:RESP:
			Valor de 0x00 a 0x04 que determina el id del servo del que 
			se requirio la posicion. Valor entre 0x00 y 0xB4 que representa
			el rango de 0 a 180 con 1 de presicion.
			*/
			i = ((cmd->data)[0] & 0x07); // Servo destinatario
			if (i < 5)
			{
				response.data[0] = pos[i];
				response.len++;
			}	
		break;
 		case SERVO_MOTOR_GET_ALL_POSITIONS:
			/* Obtiene las �ltimas posiciones de todos los servomotor
			conectados al controlador.
			:DATO:
			-
			:RESP:
			Consta de 5 valores entre 0x00 y 0xB4 concatenados, uno para 
			cada uno de los servos conectados al controlador. Cada valor 
			representa el rango de 0 a 180 con 1 de presicion.
			*/
			response.data[0] = pos[0];
			response.data[1] = pos[1];
			response.data[2] = pos[2];
			response.data[3] = pos[3];
			response.data[4] = pos[4];
			response.len += 5;
		break;
		case SERVO_MOTOR_SET_SERVO_SPEED:
			/* Determina la velocidad a la que el servomotor indicado 
			llegara a la posicion.
			:DATO:
			Valor de 0x00 a 0x04 que determina el id del servo al que 
			se le aplicara la velocidad. Valor entre 0x00 y 0xB4, 
			velocidad en grados por segundo.
			:RESP:
			-
			*/
		break;
 		case SERVO_MOTOR_SET_ALL_SPEEDS:
			/* Determina las velocidades a la que cada uno de los 
			servomotores llegara a la posicion indicada.
			:DATO:
			Consta de 5 valores entre 0x00 y 0xB4 concatenados, uno 
			para cada uno de los servos conectados al controlador. 
			Cada valor representa a la velocidad en grados por segundo.
			:RESP:
			-
			*/
		break;
 		case SERVO_MOTOR_GET_SERVO_SPEED:
			/* Obtiene la velocidad asignada al servomotor indicado.
			:DATO:
			Valor de 0x00 a 0x04 que determina el id del servo del que
			se requiere la velocidad.
			:RESP:
			Valor de 0x00 a 0x04 que determina el id del servo del que 
			se requirio la velocidad. Valor entre 0x00 y 0xB4, velocidad 
			en grados por segundo.
			*/
		break;
 		case SERVO_MOTOR_GET_ALL_SPEEDS:
			/* Obtiene las velocidades de cada uno de los servomotor 
			conectados al controlador.
			:DATO:
			-
			:RESP:
			Consta de 5 valores entre 0x00 y 0xB4 concatenados, uno para 
			cada uno de los servos conectados al controlador. Cada valor 
			representa a la velocidad en grados por segundo.
			*/
		break;
 		case SERVO_MOTOR_FREE_SERVO:
			/* Deja de aplicar fuerza sobre el servo indicado.
			:DATO:
			Valor de 0x00 a 0x04 que determina el id del servo a liberar.
			:RESP:
			-
			*/
			i = ((cmd->data)[0] & 0x07); // Servo destinatario
			if ((i < 5) && (i >= 0))
			{
				servo[i] = 0;
			}
		break;
 		case SERVO_MOTOR_FREE_ALL_SERVOS:
			/* Deja de aplicar fuerza sobre cada uno de los servomotor 
			conectados al controlador.
			:DATO:
			-
			:RESP:
			-
			*/
			servo[0] = 0;
			servo[1] = 0;
			servo[2] = 0;
			servo[3] = 0;
			servo[4] = 0;
		break;
		default:
			response.len++;
			response.cmd = COMMON_ERROR;
			response.data[0] = 0x01; // Comando desconocido
		break;
	}	

	// CRC de la respuesta
	response.crc = generate_8bit_crc((char *)(&response), response.len, CRC_PATTERN);

	return;
}
\end{verbatimtab}
}

\newpage

\section{Costo del prototipo}
\label{hA_costo}

Principales costos de armado del prototipo.

\begin{table}[h]
	\begin{center}
		\begin{tabular}{|c|l|r|r|}
			\hline
			Cantidad & Detalle & Unitario  & Total \\
			\hline
			1 & Netbook & 2451 & 2451 \\
			\hline
			4 & Placa gen\'erica & 97 & 388 \\
			\hline
			2 & Placa controladora motor DC & 151 & 302 \\
			\hline
			4 & Placa controladora de sensores & 106 & 424 \\
			\hline
			2 & Motorreductor IGNIS MR2-FA & 50 & 100 \\
			\hline
			10 & Tel\'emetro infrarrojo GP2D120 & 35 & 280 \\
			\hline
			1 & Sensor de ultrasonido SRF05 & 106 & 106 \\
			\hline
			5 & Sensor de piso CNY70 & 10 & 50 \\
			\hline
			2 & Rueda 100x26 & 30 & 60 \\
			\hline
			1 & Bater\'ia $12v$ $7Ah$  & 85 & 85 \\
			\hline
			10 & Microcontrolador PIC16F88 & 23 & 230 \\
			\hline
			2 & Driver L298 & 21 & 42 \\
			\hline
			1 & Adaptador USB - Serial RS232 & 90 & 90 \\
			\hline
			1 & Cargador de bater\'ia autom\'atico & 68 & 68 \\
			\hline
			1 & Rueda castor 30C & 11 & 11 \\
			\hline
			1 & Rueda castor 40C & 14 & 14 \\
			\hline
		\end{tabular}
	\end{center}
	\caption{Lista de materiales.}
	\label{hT_motorDC}
\end{table}