\section{Introducci\'on}
\label{}

\section{Requerimientos}
\label{}

\section{Ideas de implementaci\'on}
\label{}

\subsection{Locomoci\'on}
\label{}

distintos tipos de locomocion que tuvimos en cuenta y porque elegimos este

\subsection{Sensado del entorno}
\label{}

distintos tipos de sensores disponibles y porque elegimos estos

\subsection{Controlador}
\label{}

distintas formas de diagramar la forma de control, que tipo de controladores necesitamos, en cuales pensamos, con cuales nos quedamos

\subsection{M\'etodo de recolecci\'on}
\label{}

\section{Actuadores}
\label{}

tipos de actuadores elegidos y porque

\subsection{Motores de cont\'inua}
\label{}

principio de funcionamiento, ventajas y desventajas

\subsubsection{Caracter\'isticas}
\label{}

modelo, marca, medidas, alimentacion, consumo, velocidad, caja reductora, relacion de caja, torque, encoders, idx

\subsubsection{Circuito de control}
\label{}

puente H, diodos, conexion con el micro, pwm, salida del encoder al micro, timer/counter, circuito minimo (diagrama)

\subsubsection{Diagrama de conexi\'on}
\label{}

asignacion de pines

\subsubsection{Rutinas de control}
\label{}

codigo de control de velocidad (explicacion)

\subsection{Servo motores}
\label{}

principio de funcionamiento, caracteristicas principales (generales), torque!

\subsubsection{Circuito de control}
\label{}

alimentaci\'on, consumo, conexion con el micro, pwm, frecuencia de control, rosamiento libre, circuito minimo (diagrama)

\subsubsection{Diagrama de conexi\'on}
\label{}

asignacion de pines

\subsubsection{Rutinas de control}
\label{}

codigo de control de posicion (explicacion)

\section{Sensado}
\label{}

tipos de sensores elegidos y por que los elegimos

\subsection{Tel\'emetros infrarrojos}
\label{}

principio de funcionamiento

\subsubsection{Caracter\'isticas}
\label{}

modelo, marca, medidas, alimentacion, consumo, tiempo de muestreo, tipo de salida, rangos de distancia, rangos de voltaje, distancia vs voltaje

\subsubsection{Circuito de control}
\label{}

alimentacion, conmutacion (transistor, estado de habilitacion: 0), conexion con el micro, salida del sensor, modulo ADC, muestreo, capacitor para alimentacion, circuito minimo (diagrama)

\subsubsection{Diagrama de conexi\'on}
\label{}

asignacion de pines

\subsubsection{Rutinas de control}
\label{}

codigo de lectura de distancia (explicacion)

\subsection{Sensor de distancia por ultrasonido}
\label{}

principio de funcionamiento

\subsubsection{Caracter\'isticas}
\label{}

modelo, marca, medidas, alimentacion, consumo, tiempo/frecuencia de muestreo, tipo de salida, rangos de distancia, rango de ancho de pulso, distancia vs ancho del pulso

\subsubsection{Circuito de control}
\label{}

alimentacion, conmutacion (transistor, estado de habilitacion: 0), conexion con el micro, salida del sensor, modulo ADC, muestreo, capacitor para alimentacion, circuito minimo (diagrama)

\subsubsection{Diagrama de conexi\'on}
\label{}

asignacion de pines

\subsubsection{Rutinas de control}
\label{}

codigo de lectura de distancia (explicacion)

\subsection{Sensor reflectivo de piso}
\label{}

principio de funcionamiento

\subsubsection{Caracter\'isticas}
\label{}

modelo, marca, medidas, alimentacion, consumo, tiempo/frecuencia de muestreo, tipo de salida, distancia optima, rangos de voltaje, distancia vs voltaje

\subsubsection{Circuito de control}
\label{}

alimentacion, resistencias elegidas, conmutacion (transistor, estado de habilitacion: 0), conexion con el micro, salida del sensor, modulo ADC, muestreo, capacitor para alimentacion, circuito minimo (diagrama)

\subsubsection{Diagrama de conexi\'on}
\label{}

asignacion de pines

\subsubsection{Rutinas de control}
\label{}

codigo de lectura de nivel de reflexion (explicacion)

\subsection{Encoders}
\label{}

principio de funcionamiento

\subsubsection{Caracter\'isticas}
\label{}

tipo de encoders, cuentas x vuelta de eje de motor, velocidad maxima y minima recomendable

\subsubsection{Circuito de control}
\label{}

alimentacion, conexionado, circuito, resistencias pull-up, swtch selector, timer/counter

\subsubsection{Diagrama de conexi\'on}
\label{}

asignacion de pines

\subsubsection{Rutinas de control}
\label{}

codigo de lectura y correcion de la velocidad del motor (explicacion)

\subsection{Sensado de la bateria}
\label{}

principio de funcionamiento

\subsubsection{Caracter\'isticas}
\label{}

grafico/tabla de voltaje bateria vs salida

\subsubsection{Circuito de control}
\label{}

conexionado, circuito, modulo ADC, muestreo

\subsubsection{Rutinas de control}
\label{}

codigo de lectura de nivel de tension en la bateria (explicacion)

\subsection{Consumo del motor}
\label{}

principio de funcionamiento

\subsubsection{Caracter\'isticas}
\label{}

grafico/tabla de corriente consumida vs voltaje, caracteristicas del puente H, valores maximos y minimos, mensajes de consumo alto

\subsubsection{Circuito de control}
\label{}

valor de la resistencia, circuito, modulo ADC, muestreo, vref en el micro

\subsubsection{Rutinas de control}
\label{}

codigo de lectura de nivel de tension en la bateria (explicacion)

\subsubsection{Pulsador u otro dispositivo disparador}
\label{}

posibilidad de poner un pulsador o cualquier otro dispositivo que genere un cambio de estado y lo detecte como trigger

\subsubsection{Circuito de control}
\label{}

alimentacion, consumo, circuito, interrupciones

\subsubsection{Rutinas de control}
\label{}

codigo de lectura de cambio de estado en el pin de trigger (explicacion)

\section{Controladores}
\label{}

\subsection{Netbook}
\label{}

modelo, marca, caracteristicas, para que se usa, sistema operativo y lenguaje de programacion

\subsection{Microcontrolador}
\label{}

para que vamos a usar el micro y sus funciones principales

\subsubsection{Caracter\'isticas}
\label{}

modelo, marca, familia, memorias, etc

\subsubsection{Diagrama del microcontrolador}
\label{}

grafico y asignacion de pines x modulo

\subsubsection{M\'odulos internos}
\label{}

listado de modulos que tiene y caracteristicas de cada uno

\subsubsection{Programaci\'on del firmware}
\label{}

pines de programacion, icd2, IDE, lenguaje, version

\section{Comunicaci\'on}
\label{}

porque necesitamos comunicar los modulos, que necesidades hay, nivel de uso

\subsection{Conectividad entre m\'odulos}
\label{}

daisy chain, diagrama, montado sobre rs232, control de errores

\subsection{Protocolo de comunicaci\'on}
\label{}

caracteristicas necesarias en el protocolo, porque es importante, cosas que tuvimos en cuenta y decisiones, control de errores

\subsubsection{Caracter\'isticas b\'asicas}
\label{}

formado por paquetes, formato basico del paquete (header), control de errores

\subsubsection{Comandos comunes}
\label{}

contelo o listado de comandos comunes (en detalle o se van a un apendice) - son pocos.

\subsubsection{Comandos espec\'ificos}
\label{}

contelo o listado de comandos especificos segun el tipo de placa (referencia a un apendice con cada uno explicado)

\subsubsection{Estad\'isticas}
\label{}

analisis de paquetes por segundo, bytes de datos vs bytes de header, retransmisiones, etc

\section{Placas controladoras}
\label{}

porque tuvimos que diseñar nuestras propias placas, cosas que tuvimos en cuenta y decisiones tomadas, codigos fuente a los apendices

\subsection{Placa gen\'erica}
\label{}

funcion de una placa generica, porque fue armada, para que sirve

\subsubsection{Caracter\'isticas principales}
\label{}

testeo de nuevos modulos, testeo de la programacion, snifear la comunicacion, futuras expansiones

\subsubsection{M\'odulo de comunicaci\'on}
\label{}

explicacion de la comunicacion, igual en todas, switch de configuracion, pines, fichas, nodos en la cadena, cables pc-placa y placa-placa, max232

\subsubsection{Alimentaci\'on de la placa}
\label{}

tension para la alimentacion, circuito de la fuente, consumo maximo, voltaje minimo de alimentacion, alimentacion de 5V directos

\subsubsection{Configuraci\'on}
\label{}

configuracion minima de la placa, leds, comunicacion, header de programacion

\subsubsection{Esquem\'atico}
\label{}

esquematicos de la placa

\subsubsection{Circuito}
\label{}

circuito de la placa

\subsubsection{C\'odigo b\'asico}
\label{}

explicacion de lo minimo que deberia tener para ser parte de la cadena de comunicacion

\subsubsection{Posibles extensiones}
\label{}

posibles extensiones a futuro de la placa - nuevos modulos de testeo o control o lectura muy basica de señales, pasar a montaje superficial los componentes, hacerla mas chica

\subsection{Placa controladora de motores DC}
\label{}

funcion de una placa controladora de motorDC, porque fue armada, para que sirve, porque hay 2, porque no esta en una sola

\subsubsection{Caracter\'isticas principales}
\label{}

principio de funcionamiento, como logra controlar la velocidad, como logra ser parte de la cadena, como logra sensar el consumo, controlar el motor, puente H, diodos, leds, VREF

\subsubsection{M\'odulo de comunicaci\'on}
\label{}

se explico en el modulo generico, se agregan los comandos especificos y se puede explicar como se obtiene la informacion para dar las respuestas

\subsubsection{Alimentaci\'on de la placa}
\label{}

se explico en el modulo generico, tension para la alimentacion para los motores, necesidad de masa unica como referencia, consumo aproximado de los motores

\subsubsection{Configuraci\'on}
\label{}

configuracion de la placa, leds, comunicacion, header de programacion, switch de seleccion de encoder

\subsubsection{Esquem\'atico}
\label{}

esquematicos de la placa

\subsubsection{Circuito}
\label{}

circuito de la placa

\subsubsection{C\'odigo b\'asico}
\label{}

explicacion de lo minimo que deberia tener para ser parte de la cadena de comunicacion, sensado y control de la velocidad de los motores

\subsubsection{Posibles extensiones}
\label{}

unificar en una placa el control de mas de un motor, pasar a montaje superficial los componentes, hacerla mas chica

\subsection{Placas de sensado}
\label{}

funcion de una placa de sensado, porque fue armada, para que sirve, que tipo de sensores puedo conectar, cuales son las posibles configuraciones, diferencias, sensado de la bateria

\subsubsection{Caracter\'isticas principales}
\label{}

principio de funcionamiento, como logra tomar las muestras de los sensores, como logra ser parte de la cadena, seteo de los tipos de sensores

\subsubsection{M\'odulo de comunicaci\'on}
\label{}

se explico en el modulo generico, se agregan los comandos especificos y se puede explicar como se obtiene la informacion para dar las respuestas

\subsubsection{Alimentaci\'on de la placa}
\label{}

se explico en el modulo generico

\subsubsection{Configuraci\'on}
\label{}

configuracion de la placa, comunicacion, header de programacion

\subsubsection{Esquem\'atico}
\label{}

esquematicos de la placa

\subsubsection{Circuito}
\label{}

circuito de la placa

\subsubsection{C\'odigo b\'asico}
\label{}

explicacion de lo minimo que deberia tener para ser parte de la cadena de comunicacion y sensado de los distintos perifericos

\subsubsection{Posibles extensiones}
\label{}

uso de componentes como resistencias variables para regular la alimentacion de los sensores de piso y resistencias pull-up, pasar a montaje superficial los componentes, hacerla mas chica

\subsection{Placa controladora de servo motores}
\label{}

funcion de una placa controladora de servos, porque no fue armada, para que se penso, alguna otra opcion de conexion, pines libres

\subsubsection{Caracter\'isticas principales}
\label{}

principio de funcionamiento, como logra generar varios pwm por software, como logra ser parte de la cadena

\subsubsection{M\'odulo de comunicaci\'on}
\label{}

se explico en el modulo generico, se agregan los comandos especificos y se puede explicar como se obtiene la informacion para dar las respuestas

\subsubsection{Alimentaci\'on de la placa}
\label{}

se explico en el modulo generico, con modificaciones que permiten que circule una mayor cantidad de corriente para alimentar a los servos.

\subsubsection{Configuraci\'on}
\label{}

configuracion de la placa, comunicacion, header de programacion

\subsubsection{Esquem\'atico}
\label{}

esquematicos de la placa

\subsubsection{Circuito}
\label{}

circuito de la placa

\subsubsection{C\'odigo b\'asico}
\label{}

explicacion de lo minimo que deberia tener para ser parte de la cadena de comunicacion y control de los servos

\subsubsection{Posibles extensiones}
\label{}

uso de componentes como resistencias variables para regular la alimentacion de los sensores de piso y resistencias pull-up, pasar a montaje superficial los componentes, hacerla mas chica

\section{Armado del prototipo}
\label{}

\subsection{Dise\~no}
\label{}

\subsection{Caracter\'isticas}
\label{}

\subsection{Desarme}
\label{}

\subsection{Costo y proveedores}
\label{}
